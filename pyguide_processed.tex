% Options for packages loaded elsewhere
\PassOptionsToPackage{unicode}{hyperref}
\PassOptionsToPackage{hyphens}{url}
%
\documentclass[
]{article}
\usepackage{amsmath,amssymb}
\usepackage{array}
\usepackage[margin=2.5cm]{geometry}
\usepackage{iftex}
\ifPDFTeX
  \usepackage[T1]{fontenc}
  \usepackage[utf8]{inputenc}
  \usepackage{textcomp} % provide euro and other symbols
\else % if luatex or xetex
  \usepackage{unicode-math} % this also loads fontspec
  \defaultfontfeatures{Scale=MatchLowercase}
  \defaultfontfeatures[\rmfamily]{Ligatures=TeX,Scale=1}
\fi
\usepackage{lmodern}
\ifPDFTeX\else
  % xetex/luatex font selection
\fi
% Use upquote if available, for straight quotes in verbatim environments
\IfFileExists{upquote.sty}{\usepackage{upquote}}{}
\IfFileExists{microtype.sty}{% use microtype if available
  \usepackage[]{microtype}
  \UseMicrotypeSet[protrusion]{basicmath} % disable protrusion for tt fonts
}{}
\makeatletter
\@ifundefined{KOMAClassName}{% if non-KOMA class
  \IfFileExists{parskip.sty}{%
    \usepackage{parskip}
  }{% else
    \setlength{\parindent}{0pt}
    \setlength{\parskip}{6pt plus 2pt minus 1pt}}
}{% if KOMA class
  \KOMAoptions{parskip=half}}
\makeatother
\usepackage{xcolor}
\usepackage{color}
\usepackage{fancyvrb}
\newcommand{\VerbBar}{|}
\newcommand{\VERB}{\Verb[commandchars=\\\{\}]}
\DefineVerbatimEnvironment{Highlighting}{Verbatim}{commandchars=\\\{\}}
% Add ',fontsize=\small' for more characters per line
\newenvironment{Shaded}{}{}
\newcommand{\AlertTok}[1]{\textcolor[rgb]{1.00,0.00,0.00}{\textbf{#1}}}
\newcommand{\AnnotationTok}[1]{\textcolor[rgb]{0.38,0.63,0.69}{\textbf{\textit{#1}}}}
\newcommand{\AttributeTok}[1]{\textcolor[rgb]{0.49,0.56,0.16}{#1}}
\newcommand{\BaseNTok}[1]{\textcolor[rgb]{0.25,0.63,0.44}{#1}}
\newcommand{\BuiltInTok}[1]{\textcolor[rgb]{0.00,0.50,0.00}{#1}}
\newcommand{\CharTok}[1]{\textcolor[rgb]{0.25,0.44,0.63}{#1}}
\newcommand{\CommentTok}[1]{\textcolor[rgb]{0.38,0.63,0.69}{\textit{#1}}}
\newcommand{\CommentVarTok}[1]{\textcolor[rgb]{0.38,0.63,0.69}{\textbf{\textit{#1}}}}
\newcommand{\ConstantTok}[1]{\textcolor[rgb]{0.53,0.00,0.00}{#1}}
\newcommand{\ControlFlowTok}[1]{\textcolor[rgb]{0.00,0.44,0.13}{\textbf{#1}}}
\newcommand{\DataTypeTok}[1]{\textcolor[rgb]{0.56,0.13,0.00}{#1}}
\newcommand{\DecValTok}[1]{\textcolor[rgb]{0.25,0.63,0.44}{#1}}
\newcommand{\DocumentationTok}[1]{\textcolor[rgb]{0.73,0.13,0.13}{\textit{#1}}}
\newcommand{\ErrorTok}[1]{\textcolor[rgb]{1.00,0.00,0.00}{\textbf{#1}}}
\newcommand{\ExtensionTok}[1]{#1}
\newcommand{\FloatTok}[1]{\textcolor[rgb]{0.25,0.63,0.44}{#1}}
\newcommand{\FunctionTok}[1]{\textcolor[rgb]{0.02,0.16,0.49}{#1}}
\newcommand{\ImportTok}[1]{\textcolor[rgb]{0.00,0.50,0.00}{\textbf{#1}}}
\newcommand{\InformationTok}[1]{\textcolor[rgb]{0.38,0.63,0.69}{\textbf{\textit{#1}}}}
\newcommand{\KeywordTok}[1]{\textcolor[rgb]{0.00,0.44,0.13}{\textbf{#1}}}
\newcommand{\NormalTok}[1]{#1}
\newcommand{\OperatorTok}[1]{\textcolor[rgb]{0.40,0.40,0.40}{#1}}
\newcommand{\OtherTok}[1]{\textcolor[rgb]{0.00,0.44,0.13}{#1}}
\newcommand{\PreprocessorTok}[1]{\textcolor[rgb]{0.74,0.48,0.00}{#1}}
\newcommand{\RegionMarkerTok}[1]{#1}
\newcommand{\SpecialCharTok}[1]{\textcolor[rgb]{0.25,0.44,0.63}{#1}}
\newcommand{\SpecialStringTok}[1]{\textcolor[rgb]{0.73,0.40,0.53}{#1}}
\newcommand{\StringTok}[1]{\textcolor[rgb]{0.25,0.44,0.63}{#1}}
\newcommand{\VariableTok}[1]{\textcolor[rgb]{0.10,0.09,0.49}{#1}}
\newcommand{\VerbatimStringTok}[1]{\textcolor[rgb]{0.25,0.44,0.63}{#1}}
\newcommand{\WarningTok}[1]{\textcolor[rgb]{0.38,0.63,0.69}{\textbf{\textit{#1}}}}
\setlength{\emergencystretch}{3em} % prevent overfull lines
\providecommand{\tightlist}{%
  \setlength{\itemsep}{0pt}\setlength{\parskip}{0pt}}
% \setcounter{secnumdepth}{-\maxdimen} % remove section numbering
\ifLuaTeX
  \usepackage{selnolig}  % disable illegal ligatures
\fi
\usepackage{bookmark}
\IfFileExists{xurl.sty}{\usepackage{xurl}}{} % add URL line breaks if available
\urlstyle{same}
\hypersetup{
  hidelinks,
  pdfcreator={LaTeX via pandoc}}

\title{Google Python Style Guide}
\date{\today}

\makeatletter
\renewcommand{\maketitle}{
    \begin{titlepage}
        \begin{center}
            \vspace*{\fill}
            {\Huge \textbf{\@title}}
            \par\vspace{1.5cm}
            {\large \@date} \\[4cm]
            \hrule
            \begin{abstract}
              \large
              The Google Python Style Guide is a comprehensive set of best practices and conventions designed to ensure consistency, readability, and maintainability in Python code.
              It covers a wide range of topics, including naming conventions, code formatting, documentation, and programming practices.
              The guide emphasizes the importance of writing clear, concise, and efficient code, and provides practical advice on how to achieve this. By adhering to the guidelines outlined in the Google Python Style Guide, developers can create code that is not only functional but also easy to understand and maintain, thereby promoting collaboration and reducing the likelihood of errors.
              This document serves as a valuable resource for both individual developers and teams looking to adopt a standardized approach to Python programming.
            \end{abstract}
            \hrule
            \vspace*{\fill}
          \end{center}
        \end{titlepage}
        }
\makeatother

\begin{document}

\maketitle

\clearpage
{
\setcounter{tocdepth}{3}
\tableofcontents
}

\clearpage
\section{Background}

Python is the main dynamic language used at Google. This style guide is
a list of \emph{dos and don'ts} for Python programs.

To help you format code correctly, we've created a
\href{google_python_style.vim}{settings file for Vim}. For Emacs, the
default settings should be fine.

Many teams use the \href{https://github.com/psf/black}{Black} or
\href{https://github.com/google/pyink}{Pyink} auto-formatter to avoid
arguing over formatting.

\section{Python Language Rules}

\subsection{Lint}

Run \texttt{pylint} over your code using this
\href{https://google.github.io/styleguide/pylintrc}{pylintrc}.

\paragraph{Definition}

\texttt{pylint} is a tool for finding bugs and style problems in Python
source code. It finds problems that are typically caught by a compiler
for less dynamic languages like C and C++. Because of the dynamic nature
of Python, some warnings may be incorrect; however, spurious warnings
should be fairly infrequent.

\paragraph{Pros}

Catches easy-to-miss errors like typos, using-vars-before-assignment,
etc.

\paragraph{Cons}

\texttt{pylint} isn't perfect. To take advantage of it, sometimes we'll
need to write around it, suppress its warnings or fix it.

\paragraph{Decision}

Make sure you run \texttt{pylint} on your code.

Suppress warnings if they are inappropriate so that other issues are not
hidden. To suppress warnings, you can set a line-level comment:

\begin{samepage}
\begin{Shaded}
\begin{Highlighting}[]
\KeywordTok{def}\NormalTok{ do\_PUT(}\VariableTok{self}\NormalTok{):  }\CommentTok{\# WSGI name, so pylint: disable=invalid{-}name}
\NormalTok{  ...}
\end{Highlighting}
\end{Shaded}
\end{samepage}

\texttt{pylint} warnings are each identified by symbolic name
(\texttt{empty-docstring}) Google-specific warnings start with
\texttt{g-}.

If the reason for the suppression is not clear from the symbolic name,
add an explanation.

Suppressing in this way has the advantage that we can easily search for
suppressions and revisit them.

You can get a list of \texttt{pylint} warnings by doing:

\begin{samepage}
\begin{Shaded}
\begin{Highlighting}[]
\NormalTok{pylint {-}{-}list{-}msgs}
\end{Highlighting}
\end{Shaded}
\end{samepage}

To get more information on a particular message, use:

\begin{samepage}
\begin{Shaded}
\begin{Highlighting}[]
\NormalTok{pylint {-}{-}help{-}msg=invalid{-}name}
\end{Highlighting}
\end{Shaded}
\end{samepage}

Prefer \texttt{pylint:\ disable} to the deprecated older form
\texttt{pylint:\ disable-msg}.

Unused argument warnings can be suppressed by deleting the variables at
the beginning of the function. Always include a comment explaining why
you are deleting it. ``Unused.'' is sufficient. For example:

\begin{samepage}
\begin{Shaded}
\begin{Highlighting}[]
\KeywordTok{def}\NormalTok{ viking\_cafe\_order(spam: }\BuiltInTok{str}\NormalTok{, beans: }\BuiltInTok{str}\NormalTok{, eggs: }\BuiltInTok{str} \OperatorTok{|} \VariableTok{None} \OperatorTok{=} \VariableTok{None}\NormalTok{) }\OperatorTok{{-}\textgreater{}} \BuiltInTok{str}\NormalTok{:}
    \KeywordTok{del}\NormalTok{ beans, eggs  }\CommentTok{\# Unused by vikings.}
    \ControlFlowTok{return}\NormalTok{ spam }\OperatorTok{+}\NormalTok{ spam }\OperatorTok{+}\NormalTok{ spam}
\end{Highlighting}
\end{Shaded}
\end{samepage}

Other common forms of suppressing this warning include using
`\texttt{\_}' as the identifier for the unused argument or prefixing the
argument name with `\texttt{unused\_}', or assigning them to
`\texttt{\_}'. These forms are allowed but no longer encouraged. These
break callers that pass arguments by name and do not enforce that the
arguments are actually unused.

\subsection{Imports}

Use \texttt{import} statements for packages and modules only, not for
individual types, classes, or functions.

\paragraph{Definition}

Reusability mechanism for sharing code from one module to another.

\paragraph{Pros}

The namespace management convention is simple. The source of each
identifier is indicated in a consistent way; \texttt{x.Obj} says that
object \texttt{Obj} is defined in module \texttt{x}.

\paragraph{Cons}

Module names can still collide. Some module names are inconveniently
long.

\paragraph{Decision}

\begin{itemize}
\tightlist
\item
  Use \texttt{import\ x} for importing packages and modules.
\item
  Use \texttt{from\ x\ import\ y} where \texttt{x} is the package prefix
  and \texttt{y} is the module name with no prefix.
\item
  Use \texttt{from\ x\ import\ y\ as\ z} in any of the following
  circumstances:

  \begin{itemize}
  \tightlist
  \item
    Two modules named \texttt{y} are to be imported.
  \item
    \texttt{y} conflicts with a top-level name defined in the current
    module.
  \item
    \texttt{y} conflicts with a common parameter name that is part of
    the public API (e.g., \texttt{features}).
  \item
    \texttt{y} is an inconveniently long name.
  \item
    \texttt{y} is too generic in the context of your code (e.g.,
    \texttt{from\ \ \ \ \ storage.file\_system\ import\ options\ as\ fs\_options}).
  \end{itemize}
\item
  Use \texttt{import\ y\ as\ z} only when \texttt{z} is a standard
  abbreviation (e.g., \texttt{import\ \ \ \ \ numpy\ as\ np}).
\end{itemize}

For example the module \texttt{sound.effects.echo} may be imported as
follows:

\begin{samepage}
\begin{Shaded}
\begin{Highlighting}[]
\ImportTok{from}\NormalTok{ sound.effects }\ImportTok{import}\NormalTok{ echo}
\NormalTok{...}
\NormalTok{echo.EchoFilter(}\BuiltInTok{input}\NormalTok{, output, delay}\OperatorTok{=}\FloatTok{0.7}\NormalTok{, atten}\OperatorTok{=}\DecValTok{4}\NormalTok{)}
\end{Highlighting}
\end{Shaded}
\end{samepage}

Do not use relative names in imports. Even if the module is in the same
package, use the full package name. This helps prevent unintentionally
importing a package twice.

\subparagraph{Exemptions}

Exemptions from this rule:

\begin{itemize}
\tightlist
\item
  Symbols from the following modules are used to support static analysis
  and type checking:

  \begin{itemize}
  \tightlist
  \item
    \hyperref[typing-imports]{\texttt{typing} module}
  \item
    \hyperref[typing-imports]{\texttt{collections.abc} module}
  \item
    \href{https://github.com/python/typing_extensions/blob/main/README.md}{\texttt{typing\_extensions}
    module}
  \end{itemize}
\item
  Redirects from the
  \href{https://six.readthedocs.io/\#module-six.moves}{six.moves
  module}.
\end{itemize}

\subsection{Packages}

Import each module using the full pathname location of the module.

\paragraph{Pros}

Avoids conflicts in module names or incorrect imports due to the module
search path not being what the author expected. Makes it easier to find
modules.

\paragraph{Cons}

Makes it harder to deploy code because you have to replicate the package
hierarchy. Not really a problem with modern deployment mechanisms.

\paragraph{Decision}

All new code should import each module by its full package name.

Imports should be as follows:

\begin{samepage}
\begin{Shaded}
\begin{Highlighting}[]
\NormalTok{Yes:}
  \CommentTok{\# Reference absl.flags in code with the complete name (verbose).}
  \ImportTok{import}\NormalTok{ absl.flags}
  \ImportTok{from}\NormalTok{ doctor.who }\ImportTok{import}\NormalTok{ jodie}

\NormalTok{  \_FOO }\OperatorTok{=}\NormalTok{ absl.flags.DEFINE\_string(...)}
\end{Highlighting}
\end{Shaded}
\end{samepage}

\begin{samepage}
\begin{Shaded}
\begin{Highlighting}[]
\NormalTok{Yes:}
  \CommentTok{\# Reference flags in code with just the module name (common).}
  \ImportTok{from}\NormalTok{ absl }\ImportTok{import}\NormalTok{ flags}
  \ImportTok{from}\NormalTok{ doctor.who }\ImportTok{import}\NormalTok{ jodie}

\NormalTok{  \_FOO }\OperatorTok{=}\NormalTok{ flags.DEFINE\_string(...)}
\end{Highlighting}
\end{Shaded}
\end{samepage}

\emph{(assume this file lives in \texttt{doctor/who/} where
\texttt{jodie.py} also exists)}

\begin{samepage}
\begin{Shaded}
\begin{Highlighting}[]
\NormalTok{No:}
  \CommentTok{\# Unclear what module the author wanted and what will be imported.  The actual}
  \CommentTok{\# import behavior depends on external factors controlling sys.path.}
  \CommentTok{\# Which possible jodie module did the author intend to import?}
  \ImportTok{import}\NormalTok{ jodie}
\end{Highlighting}
\end{Shaded}
\end{samepage}

The directory the main binary is located in should not be assumed to be
in \texttt{sys.path} despite that happening in some environments. This
being the case, code should assume that \texttt{import\ jodie} refers to
a third-party or top-level package named \texttt{jodie}, not a local
\texttt{jodie.py}.

\subsection{Exceptions}

Exceptions are allowed but must be used carefully.

\paragraph{Definition}

Exceptions are a means of breaking out of normal control flow to handle
errors or other exceptional conditions.

\paragraph{Pros}

The control flow of normal operation code is not cluttered by
error-handling code. It also allows the control flow to skip multiple
frames when a certain condition occurs, e.g., returning from N nested
functions in one step instead of having to plumb error codes through.

\paragraph{Cons}

May cause the control flow to be confusing. Easy to miss error cases
when making library calls.

\paragraph{Decision}

Exceptions must follow certain conditions:

\begin{itemize}
\item
  Make use of built-in exception classes when it makes sense. For
  example, raise a \texttt{ValueError} to indicate a programming mistake
  like a violated precondition, such as may happen when validating
  function arguments.
\item
  Do not use \texttt{assert} statements in place of conditionals or
  validating preconditions. They must not be critical to the application
  logic. A litmus test would be that the \texttt{assert} could be
  removed without breaking the code. \texttt{assert} conditionals are
  \href{https://docs.python.org/3/reference/simple_stmts.html\#the-assert-statement}{not
  guaranteed} to be evaluated. For \href{https://pytest.org}{pytest}
  based tests, \texttt{assert} is okay and expected to verify
  expectations. For example:

\begin{samepage}
  \begin{Shaded}
\begin{Highlighting}[]
\NormalTok{Yes:}
  \KeywordTok{def}\NormalTok{ connect\_to\_next\_port(}\VariableTok{self}\NormalTok{, minimum: }\BuiltInTok{int}\NormalTok{) }\OperatorTok{{-}\textgreater{}} \BuiltInTok{int}\NormalTok{:}
    \CommentTok{"""Connects to the next available port.}

\CommentTok{    Args:}
\CommentTok{      minimum: A port value greater or equal to 1024.}

\CommentTok{    Returns:}
\CommentTok{      The new minimum port.}

\CommentTok{    Raises:}
\CommentTok{      ConnectionError: If no available port is found.}
\CommentTok{    """}
    \ControlFlowTok{if}\NormalTok{ minimum }\OperatorTok{\textless{}} \DecValTok{1024}\NormalTok{:}
      \CommentTok{\# Note that this raising of ValueError is not mentioned in the doc}
      \CommentTok{\# string\textquotesingle{}s "Raises:" section because it is not appropriate to}
      \CommentTok{\# guarantee this specific behavioral reaction to API misuse.}
      \ControlFlowTok{raise} \PreprocessorTok{ValueError}\NormalTok{(}\SpecialStringTok{f\textquotesingle{}Min. port must be at least 1024, not }\SpecialCharTok{\{}\NormalTok{minimum}\SpecialCharTok{\}}\SpecialStringTok{.\textquotesingle{}}\NormalTok{)}
\NormalTok{    port }\OperatorTok{=} \VariableTok{self}\NormalTok{.\_find\_next\_open\_port(minimum)}
    \ControlFlowTok{if}\NormalTok{ port }\KeywordTok{is} \VariableTok{None}\NormalTok{:}
      \ControlFlowTok{raise} \PreprocessorTok{ConnectionError}\NormalTok{(}
          \SpecialStringTok{f\textquotesingle{}Could not connect to service on port }\SpecialCharTok{\{}\NormalTok{minimum}\SpecialCharTok{\}}\SpecialStringTok{ or higher.\textquotesingle{}}\NormalTok{)}
    \CommentTok{\# The code does not depend on the result of this assert.}
    \ControlFlowTok{assert}\NormalTok{ port }\OperatorTok{\textgreater{}=}\NormalTok{ minimum, (}
        \SpecialStringTok{f\textquotesingle{}Unexpected port }\SpecialCharTok{\{}\NormalTok{port}\SpecialCharTok{\}}\SpecialStringTok{ when minimum was }\SpecialCharTok{\{}\NormalTok{minimum}\SpecialCharTok{\}}\SpecialStringTok{.\textquotesingle{}}\NormalTok{)}
    \ControlFlowTok{return}\NormalTok{ port}
\end{Highlighting}
\end{Shaded}
\end{samepage}

\begin{samepage}
\begin{Shaded}
\begin{Highlighting}[]
\NormalTok{No:}
  \KeywordTok{def}\NormalTok{ connect\_to\_next\_port(}\VariableTok{self}\NormalTok{, minimum: }\BuiltInTok{int}\NormalTok{) }\OperatorTok{{-}\textgreater{}} \BuiltInTok{int}\NormalTok{:}
    \CommentTok{"""Connects to the next available port.}

\CommentTok{    Args:}
\CommentTok{      minimum: A port value greater or equal to 1024.}

\CommentTok{    Returns:}
\CommentTok{      The new minimum port.}
\CommentTok{    """}
    \ControlFlowTok{assert}\NormalTok{ minimum }\OperatorTok{\textgreater{}=} \DecValTok{1024}\NormalTok{, }\StringTok{\textquotesingle{}Minimum port must be at least 1024.\textquotesingle{}}
    \CommentTok{\# The following code depends on the previous assert.}
\NormalTok{    port }\OperatorTok{=} \VariableTok{self}\NormalTok{.\_find\_next\_open\_port(minimum)}
    \ControlFlowTok{assert}\NormalTok{ port }\KeywordTok{is} \KeywordTok{not} \VariableTok{None}
    \CommentTok{\# The type checking of the return statement relies on the assert.}
    \ControlFlowTok{return}\NormalTok{ port}
\end{Highlighting}
\end{Shaded}
\end{samepage}
\item
  Libraries or packages may define their own exceptions. When doing so
  they must inherit from an existing exception class. Exception names
  should end in \texttt{Error} and should not introduce repetition
  (\texttt{foo.FooError}).
\item
  Never use catch-all \texttt{except:} statements, or catch
  \texttt{Exception} or \texttt{StandardError}, unless you are

  \begin{itemize}
  \tightlist
  \item
    re-raising the exception, or
  \item
    creating an isolation point in the program where exceptions are not
    propagated but are recorded and suppressed instead, such as
    protecting a thread from crashing by guarding its outermost block.
  \end{itemize}

  Python is very tolerant in this regard and \texttt{except:} will
  really catch everything including misspelled names, sys.exit() calls,
  Ctrl+C interrupts, unittest failures and all kinds of other exceptions
  that you simply don't want to catch.
\item
  Minimize the amount of code in a \texttt{try}/\texttt{except} block.
  The larger the body of the \texttt{try}, the more likely that an
  exception will be raised by a line of code that you didn't expect to
  raise an exception. In those cases, the \texttt{try}/\texttt{except}
  block hides a real error.
\item
  Use the \texttt{finally} clause to execute code whether or not an
  exception is raised in the \texttt{try} block. This is often useful
  for cleanup, i.e., closing a file.
\end{itemize}

\subsection{Mutable} Global State

Avoid mutable global state.

\paragraph{Definition}

Module-level values or class attributes that can get mutated during
program execution.

\paragraph{Pros}

Occasionally useful.

\paragraph{Cons}

\begin{itemize}
\item
  Breaks encapsulation: Such design can make it hard to achieve valid
  objectives. For example, if global state is used to manage a database
  connection, then connecting to two different databases at the same
  time (such as for computing differences during a migration) becomes
  difficult. Similar problems easily arise with global registries.
\item
  Has the potential to change module behavior during the import, because
  assignments to global variables are done when the module is first
  imported.
\end{itemize}

\paragraph{Decision}

Avoid mutable global state.

In those rare cases where using global state is warranted, mutable
global entities should be declared at the module level or as a class
attribute and made internal by prepending an \texttt{\_} to the name. If
necessary, external access to mutable global state must be done through
public functions or class methods. See \hyperref[s3.16-naming]{Naming}
below. Please explain the design reasons why mutable global state is
being used in a comment or a doc linked to from a comment.

Module-level constants are permitted and encouraged. For example:
\texttt{\_MAX\_HOLY\_HANDGRENADE\_COUNT\ =\ 3} for an internal use
constant or \texttt{SIR\_LANCELOTS\_FAVORITE\_COLOR\ =\ "blue"} for a
public API constant. Constants must be named using all caps with
underscores. See \hyperref[s3.16-naming]{Naming} below.

\subsection{Nested/Local/Inner Classes and Functions}

Nested local functions or classes are fine when used to close over a
local variable. Inner classes are fine.

\paragraph{Definition}

A class can be defined inside of a method, function, or class. A
function can be defined inside a method or function. Nested functions
have read-only access to variables defined in enclosing scopes.

\paragraph{Pros}

Allows definition of utility classes and functions that are only used
inside of a very limited scope. Very
\href{https://en.wikipedia.org/wiki/Abstract_data_type}{ADT}-y. Commonly
used for implementing decorators.

\paragraph{Cons}

Nested functions and classes cannot be directly tested. Nesting can make
the outer function longer and less readable.

\paragraph{Decision}

They are fine with some caveats. Avoid nested functions or classes
except when closing over a local value other than \texttt{self} or
\texttt{cls}. Do not nest a function just to hide it from users of a
module. Instead, prefix its name with an \_ at the module level so that
it can still be accessed by tests.

\subsection{Comprehensions \& Generator Expressions}

Okay to use for simple cases.

\paragraph{Definition}

List, Dict, and Set comprehensions as well as generator expressions
provide a concise and efficient way to create container types and
iterators without resorting to the use of traditional loops,
\texttt{map()}, \texttt{filter()}, or \texttt{lambda}.

\paragraph{Pros}

Simple comprehensions can be clearer and simpler than other dict, list,
or set creation techniques. Generator expressions can be very efficient,
since they avoid the creation of a list entirely.

\paragraph{Cons}

Complicated comprehensions or generator expressions can be hard to read.

\paragraph{Decision}

Comprehensions are allowed, however multiple \texttt{for} clauses or
filter expressions are not permitted. Optimize for readability, not
conciseness.

\begin{samepage}
\begin{Shaded}
\begin{Highlighting}[]
\NormalTok{Yes:}
\NormalTok{  result }\OperatorTok{=}\NormalTok{ [mapping\_expr }\ControlFlowTok{for}\NormalTok{ value }\KeywordTok{in}\NormalTok{ iterable }\ControlFlowTok{if}\NormalTok{ filter\_expr]}

\NormalTok{  result }\OperatorTok{=}\NormalTok{ [}
\NormalTok{      is\_valid(metric}\OperatorTok{=}\NormalTok{\{}\StringTok{\textquotesingle{}key\textquotesingle{}}\NormalTok{: value\})}
      \ControlFlowTok{for}\NormalTok{ value }\KeywordTok{in}\NormalTok{ interesting\_iterable}
      \ControlFlowTok{if}\NormalTok{ a\_longer\_filter\_expression(value)}
\NormalTok{  ]}

\NormalTok{  descriptive\_name }\OperatorTok{=}\NormalTok{ [}
\NormalTok{      transform(\{}\StringTok{\textquotesingle{}key\textquotesingle{}}\NormalTok{: key, }\StringTok{\textquotesingle{}value\textquotesingle{}}\NormalTok{: value\}, color}\OperatorTok{=}\StringTok{\textquotesingle{}black\textquotesingle{}}\NormalTok{)}
      \ControlFlowTok{for}\NormalTok{ key, value }\KeywordTok{in}\NormalTok{ generate\_iterable(some\_input)}
      \ControlFlowTok{if}\NormalTok{ complicated\_condition\_is\_met(key, value)}
\NormalTok{  ]}

\NormalTok{  result }\OperatorTok{=}\NormalTok{ []}
  \ControlFlowTok{for}\NormalTok{ x }\KeywordTok{in} \BuiltInTok{range}\NormalTok{(}\DecValTok{10}\NormalTok{):}
    \ControlFlowTok{for}\NormalTok{ y }\KeywordTok{in} \BuiltInTok{range}\NormalTok{(}\DecValTok{5}\NormalTok{):}
      \ControlFlowTok{if}\NormalTok{ x }\OperatorTok{*}\NormalTok{ y }\OperatorTok{\textgreater{}} \DecValTok{10}\NormalTok{:}
\NormalTok{        result.append((x, y))}

  \ControlFlowTok{return}\NormalTok{ \{}
\NormalTok{      x: complicated\_transform(x)}
      \ControlFlowTok{for}\NormalTok{ x }\KeywordTok{in}\NormalTok{ long\_generator\_function(parameter)}
      \ControlFlowTok{if}\NormalTok{ x }\KeywordTok{is} \KeywordTok{not} \VariableTok{None}
\NormalTok{  \}}

  \ControlFlowTok{return}\NormalTok{ (x}\OperatorTok{**}\DecValTok{2} \ControlFlowTok{for}\NormalTok{ x }\KeywordTok{in} \BuiltInTok{range}\NormalTok{(}\DecValTok{10}\NormalTok{))}

\NormalTok{  unique\_names }\OperatorTok{=}\NormalTok{ \{user.name }\ControlFlowTok{for}\NormalTok{ user }\KeywordTok{in}\NormalTok{ users }\ControlFlowTok{if}\NormalTok{ user }\KeywordTok{is} \KeywordTok{not} \VariableTok{None}\NormalTok{\}}
\end{Highlighting}
\end{Shaded}
\end{samepage}

\begin{samepage}
\begin{Shaded}
\begin{Highlighting}[]
\NormalTok{No:}
\NormalTok{  result }\OperatorTok{=}\NormalTok{ [(x, y) }\ControlFlowTok{for}\NormalTok{ x }\KeywordTok{in} \BuiltInTok{range}\NormalTok{(}\DecValTok{10}\NormalTok{) }\ControlFlowTok{for}\NormalTok{ y }\KeywordTok{in} \BuiltInTok{range}\NormalTok{(}\DecValTok{5}\NormalTok{) }\ControlFlowTok{if}\NormalTok{ x }\OperatorTok{*}\NormalTok{ y }\OperatorTok{\textgreater{}} \DecValTok{10}\NormalTok{]}

  \ControlFlowTok{return}\NormalTok{ (}
\NormalTok{      (x, y, z)}
      \ControlFlowTok{for}\NormalTok{ x }\KeywordTok{in} \BuiltInTok{range}\NormalTok{(}\DecValTok{5}\NormalTok{)}
      \ControlFlowTok{for}\NormalTok{ y }\KeywordTok{in} \BuiltInTok{range}\NormalTok{(}\DecValTok{5}\NormalTok{)}
      \ControlFlowTok{if}\NormalTok{ x }\OperatorTok{!=}\NormalTok{ y}
      \ControlFlowTok{for}\NormalTok{ z }\KeywordTok{in} \BuiltInTok{range}\NormalTok{(}\DecValTok{5}\NormalTok{)}
      \ControlFlowTok{if}\NormalTok{ y }\OperatorTok{!=}\NormalTok{ z}
\NormalTok{  )}
\end{Highlighting}
\end{Shaded}
\end{samepage}

\subsection{Default} Iterators and Operators

Use default iterators and operators for types that support them, like
lists, dictionaries, and files.

\paragraph{Definition}

Container types, like dictionaries and lists, define default iterators
and membership test operators (``in'' and ``not in'').

\paragraph{Pros}

The default iterators and operators are simple and efficient. They
express the operation directly, without extra method calls. A function
that uses default operators is generic. It can be used with any type
that supports the operation.

\paragraph{Cons}

You can't tell the type of objects by reading the method names (unless
the variable has type annotations). This is also an advantage.

\paragraph{Decision}

Use default iterators and operators for types that support them, like
lists, dictionaries, and files. The built-in types define iterator
methods, too. Prefer these methods to methods that return lists, except
that you should not mutate a container while iterating over it.

\begin{samepage}
\begin{Shaded}
\begin{Highlighting}[]
\NormalTok{Yes:  }\ControlFlowTok{for}\NormalTok{ key }\KeywordTok{in}\NormalTok{ adict: ...}
      \ControlFlowTok{if}\NormalTok{ obj }\KeywordTok{in}\NormalTok{ alist: ...}
      \ControlFlowTok{for}\NormalTok{ line }\KeywordTok{in}\NormalTok{ afile: ...}
      \ControlFlowTok{for}\NormalTok{ k, v }\KeywordTok{in}\NormalTok{ adict.items(): ...}
\end{Highlighting}
\end{Shaded}
\end{samepage}

\begin{samepage}
\begin{Shaded}
\begin{Highlighting}[]
\NormalTok{No:   }\ControlFlowTok{for}\NormalTok{ key }\KeywordTok{in}\NormalTok{ adict.keys(): ...}
      \ControlFlowTok{for}\NormalTok{ line }\KeywordTok{in}\NormalTok{ afile.readlines(): ...}
\end{Highlighting}
\end{Shaded}
\end{samepage}

\subsection{Generators}

Use generators as needed.

\paragraph{Definition}

A generator function returns an iterator that yields a value each time
it executes a yield statement. After it yields a value, the runtime
state of the generator function is suspended until the next value is
needed.

\paragraph{Pros}

Simpler code, because the state of local variables and control flow are
preserved for each call. A generator uses less memory than a function
that creates an entire list of values at once.

\paragraph{Cons}

Local variables in the generator will not be garbage collected until the
generator is either consumed to exhaustion or itself garbage collected.

\paragraph{Decision}

Fine. Use ``Yields:'' rather than ``Returns:'' in the docstring for
generator functions.

If the generator manages an expensive resource, make sure to force the
clean up.

A good way to do the clean up is by wrapping the generator with a
context manager \href{https://peps.python.org/pep-0533/}{PEP-0533}.

\subsection{Lambda Functions}

Okay for one-liners. Prefer generator expressions over \texttt{map()} or
\texttt{filter()} with a \texttt{lambda}.

\paragraph{Definition}

Lambdas define anonymous functions in an expression, as opposed to a
statement.

\paragraph{Pros}

Convenient.

\paragraph{Cons}

Harder to read and debug than local functions. The lack of names means
stack traces are more difficult to understand. Expressiveness is limited
because the function may only contain an expression.

\paragraph{Decision}

Lambdas are allowed. If the code inside the lambda function spans
multiple lines or is longer than 60-80 chars, it might be better to
define it as a regular \hyperref[lexical-scoping]{nested function}.

For common operations like multiplication, use the functions from the
\texttt{operator} module instead of lambda functions. For example,
prefer \texttt{operator.mul} to \texttt{lambda\ x,\ y:\ x\ *\ y}.

\subsection{Conditional Expressions}

Okay for simple cases.

\paragraph{Definition}

Conditional expressions (sometimes called a ``ternary operator'') are
mechanisms that provide a shorter syntax for if statements. For example:
\texttt{x\ =\ 1\ if\ cond\ else\ 2}.

\paragraph{Pros}

Shorter and more convenient than an if statement.

\paragraph{Cons}

May be harder to read than an if statement. The condition may be
difficult to locate if the expression is long.

\paragraph{Decision}

Okay to use for simple cases. Each portion must fit on one line:
true-expression, if-expression, else-expression. Use a complete if
statement when things get more complicated.

\begin{samepage}
\begin{Shaded}
\begin{Highlighting}[]
\NormalTok{Yes:}
\NormalTok{    one\_line }\OperatorTok{=} \StringTok{\textquotesingle{}yes\textquotesingle{}} \ControlFlowTok{if}\NormalTok{ predicate(value) }\ControlFlowTok{else} \StringTok{\textquotesingle{}no\textquotesingle{}}
\NormalTok{    slightly\_split }\OperatorTok{=}\NormalTok{ (}\StringTok{\textquotesingle{}yes\textquotesingle{}} \ControlFlowTok{if}\NormalTok{ predicate(value)}
                      \ControlFlowTok{else} \StringTok{\textquotesingle{}no, nein, nyet\textquotesingle{}}\NormalTok{)}
\NormalTok{    the\_longest\_ternary\_style\_that\_can\_be\_done }\OperatorTok{=}\NormalTok{ (}
        \StringTok{\textquotesingle{}yes, true, affirmative, confirmed, correct\textquotesingle{}}
        \ControlFlowTok{if}\NormalTok{ predicate(value)}
        \ControlFlowTok{else} \StringTok{\textquotesingle{}no, false, negative, nay\textquotesingle{}}\NormalTok{)}
\end{Highlighting}
\end{Shaded}
\end{samepage}

\begin{samepage}
\begin{Shaded}
\begin{Highlighting}[]
\NormalTok{No:}
\NormalTok{    bad\_line\_breaking }\OperatorTok{=}\NormalTok{ (}\StringTok{\textquotesingle{}yes\textquotesingle{}} \ControlFlowTok{if}\NormalTok{ predicate(value) }\ControlFlowTok{else}
                         \StringTok{\textquotesingle{}no\textquotesingle{}}\NormalTok{)}
\NormalTok{    portion\_too\_long }\OperatorTok{=}\NormalTok{ (}\StringTok{\textquotesingle{}yes\textquotesingle{}}
                        \ControlFlowTok{if}\NormalTok{ some\_long\_module.some\_long\_predicate\_function(}
\NormalTok{                            really\_long\_variable\_name)}
                        \ControlFlowTok{else} \StringTok{\textquotesingle{}no, false, negative, nay\textquotesingle{}}\NormalTok{)}
\end{Highlighting}
\end{Shaded}
\end{samepage}

\subsection{Default Argument Values}

Okay in most cases.

\paragraph{Definition}

You can specify values for variables at the end of a function's
parameter list, e.g., \texttt{def\ foo(a,\ b=0):}. If \texttt{foo} is
called with only one argument, \texttt{b} is set to 0. If it is called
with two arguments, \texttt{b} has the value of the second argument.

\paragraph{Pros}

Often you have a function that uses lots of default values, but on rare
occasions you want to override the defaults. Default argument values
provide an easy way to do this, without having to define lots of
functions for the rare exceptions. As Python does not support overloaded
methods/functions, default arguments are an easy way of ``faking'' the
overloading behavior.

\paragraph{Cons}

Default arguments are evaluated once at module load time. This may cause
problems if the argument is a mutable object such as a list or a
dictionary. If the function modifies the object (e.g., by appending an
item to a list), the default value is modified.

\paragraph{Decision}

Okay to use with the following caveat:

Do not use mutable objects as default values in the function or method
definition.

\begin{samepage}
\begin{Shaded}
\begin{Highlighting}[]
\NormalTok{Yes: }\KeywordTok{def}\NormalTok{ foo(a, b}\OperatorTok{=}\VariableTok{None}\NormalTok{):}
         \ControlFlowTok{if}\NormalTok{ b }\KeywordTok{is} \VariableTok{None}\NormalTok{:}
\NormalTok{             b }\OperatorTok{=}\NormalTok{ []}
\NormalTok{Yes: }\KeywordTok{def}\NormalTok{ foo(a, b: Sequence }\OperatorTok{|} \VariableTok{None} \OperatorTok{=} \VariableTok{None}\NormalTok{):}
         \ControlFlowTok{if}\NormalTok{ b }\KeywordTok{is} \VariableTok{None}\NormalTok{:}
\NormalTok{             b }\OperatorTok{=}\NormalTok{ []}
\NormalTok{Yes: }\KeywordTok{def}\NormalTok{ foo(a, b: Sequence }\OperatorTok{=}\NormalTok{ ()):  }\CommentTok{\# Empty tuple OK since tuples are immutable.}
\NormalTok{         ...}
\end{Highlighting}
\end{Shaded}
\end{samepage}

\begin{samepage}
\begin{Shaded}
\begin{Highlighting}[]
\ImportTok{from}\NormalTok{ absl }\ImportTok{import}\NormalTok{ flags}
\NormalTok{\_FOO }\OperatorTok{=}\NormalTok{ flags.DEFINE\_string(...)}

\NormalTok{No:  }\KeywordTok{def}\NormalTok{ foo(a, b}\OperatorTok{=}\NormalTok{[]):}
\NormalTok{         ...}
\NormalTok{No:  }\KeywordTok{def}\NormalTok{ foo(a, b}\OperatorTok{=}\NormalTok{time.time()):  }\CommentTok{\# Is \textasciigrave{}b\textasciigrave{} supposed to represent when this module was loaded?}
\NormalTok{         ...}
\NormalTok{No:  }\KeywordTok{def}\NormalTok{ foo(a, b}\OperatorTok{=}\NormalTok{\_FOO.value):  }\CommentTok{\# sys.argv has not yet been parsed...}
\NormalTok{         ...}
\NormalTok{No:  }\KeywordTok{def}\NormalTok{ foo(a, b: Mapping }\OperatorTok{=}\NormalTok{ \{\}):  }\CommentTok{\# Could still get passed to unchecked code.}
\NormalTok{         ...}
\end{Highlighting}
\end{Shaded}
\end{samepage}

\subsection{Properties}

Properties may be used to control getting or setting attributes that
require trivial computations or logic. Property implementations must
match the general expectations of regular attribute access: that they
are cheap, straightforward, and unsurprising.

\paragraph{Definition}

A way to wrap method calls for getting and setting an attribute as a
standard attribute access.

\paragraph{Pros}

\begin{itemize}
\tightlist
\item
  Allows for an attribute access and assignment API rather than
  \hyperref[getters-and-setters]{getter and setter} method calls.
\item
  Can be used to make an attribute read-only.
\item
  Allows calculations to be lazy.
\item
  Provides a way to maintain the public interface of a class when the
  internals evolve independently of class users.
\end{itemize}

\paragraph{Cons}

\begin{itemize}
\tightlist
\item
  Can hide side-effects much like operator overloading.
\item
  Can be confusing for subclasses.
\end{itemize}

\paragraph{Decision}

Properties are allowed, but, like operator overloading, should only be
used when necessary and match the expectations of typical attribute
access; follow the \hyperref[getters-and-setters]{getters and setters}
rules otherwise.

For example, using a property to simply both get and set an internal
attribute isn't allowed: there is no computation occurring, so the
property is unnecessary (\hyperref[getters-and-setters]{make the
attribute public instead}). In comparison, using a property to control
attribute access or to calculate a \emph{trivially} derived value is
allowed: the logic is simple and unsurprising.

Properties should be created with the \texttt{@property}
\hyperref[s2.17-function-and-method-decorators]{decorator}. Manually
implementing a property descriptor is considered a
\hyperref[power-features]{power feature}.

Inheritance with properties can be non-obvious. Do not use properties to
implement computations a subclass may ever want to override and extend.

\subsection{True/False Evaluations}

Use the ``implicit'' false if at all possible (with a few caveats).

\paragraph{Definition}

Python evaluates certain values as \texttt{False} when in a boolean
context. A quick ``rule of thumb'' is that all ``empty'' values are
considered false, so
\texttt{0,\ None,\ {[}{]},\ \{\},\ \textquotesingle{}\textquotesingle{}}
all evaluate as false in a boolean context.

\paragraph{Pros}

Conditions using Python booleans are easier to read and less
error-prone. In most cases, they're also faster.

\paragraph{Cons}

May look strange to C/C++ developers.

\paragraph{Decision}

Use the ``implicit'' false if possible, e.g., \texttt{if\ foo:} rather
than \texttt{if\ foo\ !=\ {[}{]}:}. There are a few caveats that you
should keep in mind though:

\begin{itemize}
\item
  Always use \texttt{if\ foo\ is\ None:} (or \texttt{is\ not\ None}) to
  check for a \texttt{None} value. E.g., when testing whether a variable
  or argument that defaults to \texttt{None} was set to some other
  value. The other value might be a value that's false in a boolean
  context!
\item
  Never compare a boolean variable to \texttt{False} using \texttt{==}.
  Use \texttt{if\ not\ x:} instead. If you need to distinguish
  \texttt{False} from \texttt{None} then chain the expressions, such as
  \texttt{if\ not\ x\ and\ x\ is\ not\ None:}.
\item
  For sequences (strings, lists, tuples), use the fact that empty
  sequences are false, so \texttt{if\ seq:} and \texttt{if\ not\ seq:}
  are preferable to \texttt{if\ len(seq):} and
  \texttt{if\ not\ len(seq):} respectively.
\item
  When handling integers, implicit false may involve more risk than
  benefit (i.e., accidentally handling \texttt{None} as 0). You may
  compare a value which is known to be an integer (and is not the result
  of \texttt{len()}) against the integer 0.

\begin{samepage}
  \begin{Shaded}
\begin{Highlighting}[]
\NormalTok{Yes: }\ControlFlowTok{if} \KeywordTok{not}\NormalTok{ users:}
         \BuiltInTok{print}\NormalTok{(}\StringTok{\textquotesingle{}no users\textquotesingle{}}\NormalTok{)}

     \ControlFlowTok{if}\NormalTok{ i }\OperatorTok{\%} \DecValTok{10} \OperatorTok{==} \DecValTok{0}\NormalTok{:}
         \VariableTok{self}\NormalTok{.handle\_multiple\_of\_ten()}

     \KeywordTok{def}\NormalTok{ f(x}\OperatorTok{=}\VariableTok{None}\NormalTok{):}
         \ControlFlowTok{if}\NormalTok{ x }\KeywordTok{is} \VariableTok{None}\NormalTok{:}
\NormalTok{             x }\OperatorTok{=}\NormalTok{ []}
\end{Highlighting}
\end{Shaded}
\end{samepage}

\begin{samepage}
\begin{Shaded}
\begin{Highlighting}[]
\NormalTok{No:  }\ControlFlowTok{if} \BuiltInTok{len}\NormalTok{(users) }\OperatorTok{==} \DecValTok{0}\NormalTok{:}
         \BuiltInTok{print}\NormalTok{(}\StringTok{\textquotesingle{}no users\textquotesingle{}}\NormalTok{)}

     \ControlFlowTok{if} \KeywordTok{not}\NormalTok{ i }\OperatorTok{\%} \DecValTok{10}\NormalTok{:}
         \VariableTok{self}\NormalTok{.handle\_multiple\_of\_ten()}

     \KeywordTok{def}\NormalTok{ f(x}\OperatorTok{=}\VariableTok{None}\NormalTok{):}
\NormalTok{         x }\OperatorTok{=}\NormalTok{ x }\KeywordTok{or}\NormalTok{ []}
\end{Highlighting}
\end{Shaded}
\end{samepage}
\item
  Note that \texttt{\textquotesingle{}0\textquotesingle{}} (i.e.,
  \texttt{0} as string) evaluates to true.
\item
  Note that Numpy arrays may raise an exception in an implicit boolean
  context. Prefer the \texttt{.size} attribute when testing emptiness of
  a \texttt{np.array} (e.g.~\texttt{if\ not\ users.size}).
\end{itemize}

\subsection{Lexical Scoping}

Okay to use.

\paragraph{Definition}

A nested Python function can refer to variables defined in enclosing
functions, but cannot assign to them. Variable bindings are resolved
using lexical scoping, that is, based on the static program text. Any
assignment to a name in a block will cause Python to treat all
references to that name as a local variable, even if the use precedes
the assignment. If a global declaration occurs, the name is treated as a
global variable.

An example of the use of this feature is:

\begin{samepage}
\begin{Shaded}
\begin{Highlighting}[]
\KeywordTok{def}\NormalTok{ get\_adder(summand1: }\BuiltInTok{float}\NormalTok{) }\OperatorTok{{-}\textgreater{}}\NormalTok{ Callable[[}\BuiltInTok{float}\NormalTok{], }\BuiltInTok{float}\NormalTok{]:}
    \CommentTok{"""Returns a function that adds numbers to a given number."""}
    \KeywordTok{def}\NormalTok{ adder(summand2: }\BuiltInTok{float}\NormalTok{) }\OperatorTok{{-}\textgreater{}} \BuiltInTok{float}\NormalTok{:}
        \ControlFlowTok{return}\NormalTok{ summand1 }\OperatorTok{+}\NormalTok{ summand2}

    \ControlFlowTok{return}\NormalTok{ adder}
\end{Highlighting}
\end{Shaded}
\end{samepage}

\paragraph{Pros}

Often results in clearer, more elegant code. Especially comforting to
experienced Lisp and Scheme (and Haskell and ML and \ldots) programmers.

\paragraph{Cons}

Can lead to confusing bugs, such as this example based on
\href{https://peps.python.org/pep-0227/}{PEP-0227}:

\begin{samepage}
\begin{Shaded}
\begin{Highlighting}[]
\NormalTok{i }\OperatorTok{=} \DecValTok{4}
\KeywordTok{def}\NormalTok{ foo(x: Iterable[}\BuiltInTok{int}\NormalTok{]):}
    \KeywordTok{def}\NormalTok{ bar():}
        \BuiltInTok{print}\NormalTok{(i, end}\OperatorTok{=}\StringTok{\textquotesingle{}\textquotesingle{}}\NormalTok{)}
    \CommentTok{\# ...}
    \CommentTok{\# A bunch of code here}
    \CommentTok{\# ...}
    \ControlFlowTok{for}\NormalTok{ i }\KeywordTok{in}\NormalTok{ x:  }\CommentTok{\# Ah, i *is* local to foo, so this is what bar sees}
        \BuiltInTok{print}\NormalTok{(i, end}\OperatorTok{=}\StringTok{\textquotesingle{}\textquotesingle{}}\NormalTok{)}
\NormalTok{    bar()}
\end{Highlighting}
\end{Shaded}
\end{samepage}

So \texttt{foo({[}1,\ 2,\ 3{]})} will print \texttt{1\ 2\ 3\ 3}, not
\texttt{1\ 2\ 3\ 4}.

\paragraph{Decision}

Okay to use.

\subsection{Function and Method Decorators}

Use decorators judiciously when there is a clear advantage. Avoid
\texttt{staticmethod} and limit use of \texttt{classmethod}.

\paragraph{Definition}

\href{https://docs.python.org/3/glossary.html\#term-decorator}{Decorators
for Functions and Methods} (a.k.a ``the \texttt{@} notation''). One
common decorator is \texttt{@property}, used for converting ordinary
methods into dynamically computed attributes. However, the decorator
syntax allows for user-defined decorators as well. Specifically, for
some function \texttt{my\_decorator}, this:

\begin{samepage}
\begin{Shaded}
\begin{Highlighting}[]
\KeywordTok{class}\NormalTok{ C:}
    \AttributeTok{@my\_decorator}
    \KeywordTok{def}\NormalTok{ method(}\VariableTok{self}\NormalTok{):}
        \CommentTok{\# method body ...}
\end{Highlighting}
\end{Shaded}
\end{samepage}

is equivalent to:

\begin{samepage}
\begin{Shaded}
\begin{Highlighting}[]
\KeywordTok{class}\NormalTok{ C:}
    \KeywordTok{def}\NormalTok{ method(}\VariableTok{self}\NormalTok{):}
        \CommentTok{\# method body ...}
\NormalTok{    method }\OperatorTok{=}\NormalTok{ my\_decorator(method)}
\end{Highlighting}
\end{Shaded}
\end{samepage}

\paragraph{Pros}

Elegantly specifies some transformation on a method; the transformation
might eliminate some repetitive code, enforce invariants, etc.

\paragraph{Cons}

Decorators can perform arbitrary operations on a function's arguments or
return values, resulting in surprising implicit behavior. Additionally,
decorators execute at object definition time. For module-level objects
(classes, module functions, \ldots) this happens at import time.
Failures in decorator code are pretty much impossible to recover from.

\paragraph{Decision}

Use decorators judiciously when there is a clear advantage. Decorators
should follow the same import and naming guidelines as functions.
Decorator pydoc should clearly state that the function is a decorator.
Write unit tests for decorators.

Avoid external dependencies in the decorator itself (e.g.~don't rely on
files, sockets, database connections, etc.), since they might not be
available when the decorator runs (at import time, perhaps from
\texttt{pydoc} or other tools). A decorator that is called with valid
parameters should (as much as possible) be guaranteed to succeed in all
cases.

Decorators are a special case of ``top-level code'' - see
\hyperref[s3.17-main]{main} for more discussion.

Never use \texttt{staticmethod} unless forced to in order to integrate
with an API defined in an existing library. Write a module-level
function instead.

Use \texttt{classmethod} only when writing a named constructor, or a
class-specific routine that modifies necessary global state such as a
process-wide cache.

\subsection{Threading}

Do not rely on the atomicity of built-in types.

While Python's built-in data types such as dictionaries appear to have
atomic operations, there are corner cases where they aren't atomic
(e.g.~if \texttt{\_\_hash\_\_} or \texttt{\_\_eq\_\_} are implemented as
Python methods) and their atomicity should not be relied upon. Neither
should you rely on atomic variable assignment (since this in turn
depends on dictionaries).

Use the \texttt{queue} module's \texttt{Queue} data type as the
preferred way to communicate data between threads. Otherwise, use the
\texttt{threading} module and its locking primitives. Prefer condition
variables and \texttt{threading.Condition} instead of using lower-level
locks.

\subsection{Power} Features

Avoid these features.

\paragraph{Definition}

Python is an extremely flexible language and gives you many fancy
features such as custom metaclasses, access to bytecode, on-the-fly
compilation, dynamic inheritance, object reparenting, import hacks,
reflection (e.g.~some uses of \texttt{getattr()}), modification of
system internals, \texttt{\_\_del\_\_} methods implementing customized
cleanup, etc.

\paragraph{Pros}

These are powerful language features. They can make your code more
compact.

\paragraph{Cons}

It's very tempting to use these ``cool'' features when they're not
absolutely necessary. It's harder to read, understand, and debug code
that's using unusual features underneath. It doesn't seem that way at
first (to the original author), but when revisiting the code, it tends
to be more difficult than code that is longer but is straightforward.

\paragraph{Decision}

Avoid these features in your code.

Standard library modules and classes that internally use these features
are okay to use (for example, \texttt{abc.ABCMeta},
\texttt{dataclasses}, and \texttt{enum}).

\subsection{Modern} Python: from \_\_future\_\_ imports

New language version semantic changes may be gated behind a special
future import to enable them on a per-file basis within earlier
runtimes.

\paragraph{Definition}

Being able to turn on some of the more modern features via
\texttt{from\ \_\_future\_\_\ import} statements allows early use of
features from expected future Python versions.

\paragraph{Pros}

This has proven to make runtime version upgrades smoother as changes can
be made on a per-file basis while declaring compatibility and preventing
regressions within those files. Modern code is more maintainable as it
is less likely to accumulate technical debt that will be problematic
during future runtime upgrades.

\paragraph{Cons}

Such code may not work on very old interpreter versions prior to the
introduction of the needed future statement. The need for this is more
common in projects supporting an extremely wide variety of environments.

\paragraph{Decision}

\subparagraph{from \_\_future\_\_
imports}\label{from-__future__-imports}

Use of \texttt{from\ \_\_future\_\_\ import} statements is encouraged.
It allows a given source file to start using more modern Python syntax
features today. Once you no longer need to run on a version where the
features are hidden behind a \texttt{\_\_future\_\_} import, feel free
to remove those lines.

In code that may execute on versions as old as 3.5 rather than
\textgreater= 3.7, import:

\begin{samepage}
\begin{Shaded}
\begin{Highlighting}[]
\ImportTok{from}\NormalTok{ \_\_future\_\_ }\ImportTok{import}\NormalTok{ generator\_stop}
\end{Highlighting}
\end{Shaded}
\end{samepage}

For more information read the
\href{https://docs.python.org/3/library/__future__.html}{Python future
statement definitions} documentation.

Please don't remove these imports until you are confident the code is
only ever used in a sufficiently modern environment. Even if you do not
currently use the feature a specific future import enables in your code
today, keeping it in place in the file prevents later modifications of
the code from inadvertently depending on the older behavior.

Use other \texttt{from\ \_\_future\_\_} import statements as you see
fit.

\subsection{Type Annotated Code}

You can annotate Python code with type hints according to
\href{https://peps.python.org/pep-0484/}{PEP-484}, and type-check the
code at build time with a type checking tool like
\href{https://github.com/google/pytype}{pytype}.

Type annotations can be in the source or in a
\href{https://peps.python.org/pep-0484/\#stub-files}{stub pyi file}.
Whenever possible, annotations should be in the source. Use pyi files
for third-party or extension modules.

\paragraph{Definition}

Type annotations (or ``type hints'') are for function or method
arguments and return values:

\begin{samepage}
\begin{Shaded}
\begin{Highlighting}[]
\KeywordTok{def}\NormalTok{ func(a: }\BuiltInTok{int}\NormalTok{) }\OperatorTok{{-}\textgreater{}} \BuiltInTok{list}\NormalTok{[}\BuiltInTok{int}\NormalTok{]:}
\end{Highlighting}
\end{Shaded}
\end{samepage}

You can also declare the type of a variable using similar
\href{https://peps.python.org/pep-0526/}{PEP-526} syntax:

\begin{samepage}
\begin{Shaded}
\begin{Highlighting}[]
\NormalTok{a: SomeType }\OperatorTok{=}\NormalTok{ some\_func()}
\end{Highlighting}
\end{Shaded}
\end{samepage}

\paragraph{Pros}

Type annotations improve the readability and maintainability of your
code. The type checker will convert many runtime errors to build-time
errors, and reduce your ability to use \hyperref[power-features]{Power
Features}.

\paragraph{Cons}

You will have to keep the type declarations up to date. You might see
type errors that you think are valid code. Use of a
\href{https://github.com/google/pytype}{type checker} may reduce your
ability to use \hyperref[power-features]{Power Features}.

\paragraph{Decision}

You are strongly encouraged to enable Python type analysis when updating
code. When adding or modifying public APIs, include type annotations and
enable checking via pytype in the build system. As static analysis is
relatively new to Python, we acknowledge that undesired side-effects
(such as wrongly inferred types) may prevent adoption by some projects.
In those situations, authors are encouraged to add a comment with a TODO
or link to a bug describing the issue(s) currently preventing type
annotation adoption in the BUILD file or in the code itself as
appropriate.

\section{Python Style Rules}

\subsection{Semicolons}

Do not terminate your lines with semicolons, and do not use semicolons
to put two statements on the same line.

\subsection{Line} length

Maximum line length is \emph{80 characters}.

Explicit exceptions to the 80 character limit:

\begin{itemize}
\tightlist
\item
  Long import statements.
\item
  URLs, pathnames, or long flags in comments.
\item
  Long string module-level constants not containing whitespace that
  would be inconvenient to split across lines such as URLs or pathnames.

  \begin{itemize}
  \tightlist
  \item
    Pylint disable comments. (e.g.:
    \texttt{\#\ pylint:\ disable=invalid-name})
  \end{itemize}
\end{itemize}

Do not use a backslash for
\href{https://docs.python.org/3/reference/lexical_analysis.html\#explicit-line-joining}{explicit
line continuation}.

Instead, make use of Python's
\href{http://docs.python.org/reference/lexical_analysis.html\#implicit-line-joining}{implicit
line joining inside parentheses, brackets and braces}. If necessary, you
can add an extra pair of parentheses around an expression.

Note that this rule doesn't prohibit backslash-escaped newlines within
strings (see \hyperref[strings]{below}).

\begin{samepage}
\begin{Shaded}
\begin{Highlighting}[]
\NormalTok{Yes: foo\_bar(}\VariableTok{self}\NormalTok{, width, height, color}\OperatorTok{=}\StringTok{\textquotesingle{}black\textquotesingle{}}\NormalTok{, design}\OperatorTok{=}\VariableTok{None}\NormalTok{, x}\OperatorTok{=}\StringTok{\textquotesingle{}foo\textquotesingle{}}\NormalTok{,}
\NormalTok{             emphasis}\OperatorTok{=}\VariableTok{None}\NormalTok{, highlight}\OperatorTok{=}\DecValTok{0}\NormalTok{)}
\end{Highlighting}
\end{Shaded}
\end{samepage}

\begin{samepage}
\begin{Shaded}
\begin{Highlighting}[]

\NormalTok{Yes: }\ControlFlowTok{if}\NormalTok{ (width }\OperatorTok{==} \DecValTok{0} \KeywordTok{and}\NormalTok{ height }\OperatorTok{==} \DecValTok{0} \KeywordTok{and}
\NormalTok{         color }\OperatorTok{==} \StringTok{\textquotesingle{}red\textquotesingle{}} \KeywordTok{and}\NormalTok{ emphasis }\OperatorTok{==} \StringTok{\textquotesingle{}strong\textquotesingle{}}\NormalTok{):}

\NormalTok{     (bridge\_questions.clarification\_on}
\NormalTok{      .average\_airspeed\_of.unladen\_swallow) }\OperatorTok{=} \StringTok{\textquotesingle{}African or European?\textquotesingle{}}

     \ControlFlowTok{with}\NormalTok{ (}
\NormalTok{         very\_long\_first\_expression\_function() }\ImportTok{as}\NormalTok{ spam,}
\NormalTok{         very\_long\_second\_expression\_function() }\ImportTok{as}\NormalTok{ beans,}
\NormalTok{         third\_thing() }\ImportTok{as}\NormalTok{ eggs,}
\NormalTok{     ):}
\NormalTok{       place\_order(eggs, beans, spam, beans)}
\end{Highlighting}
\end{Shaded}
\end{samepage}

\begin{samepage}
\begin{Shaded}
\begin{Highlighting}[]

\NormalTok{No:  }\ControlFlowTok{if}\NormalTok{ width }\OperatorTok{==} \DecValTok{0} \KeywordTok{and}\NormalTok{ height }\OperatorTok{==} \DecValTok{0} \KeywordTok{and} \OperatorTok{\textbackslash{}}
\NormalTok{         color }\OperatorTok{==} \StringTok{\textquotesingle{}red\textquotesingle{}} \KeywordTok{and}\NormalTok{ emphasis }\OperatorTok{==} \StringTok{\textquotesingle{}strong\textquotesingle{}}\NormalTok{:}

\NormalTok{     bridge\_questions.clarification\_on }\OperatorTok{\textbackslash{}}
\NormalTok{         .average\_airspeed\_of.unladen\_swallow }\OperatorTok{=} \StringTok{\textquotesingle{}African or European?\textquotesingle{}}

     \ControlFlowTok{with}\NormalTok{ very\_long\_first\_expression\_function() }\ImportTok{as}\NormalTok{ spam, }\OperatorTok{\textbackslash{}}
\NormalTok{           very\_long\_second\_expression\_function() }\ImportTok{as}\NormalTok{ beans, }\OperatorTok{\textbackslash{}}
\NormalTok{           third\_thing() }\ImportTok{as}\NormalTok{ eggs:}
\NormalTok{       place\_order(eggs, beans, spam, beans)}
\end{Highlighting}
\end{Shaded}
\end{samepage}

When a literal string won't fit on a single line, use parentheses for
implicit line joining.

\begin{samepage}
\begin{Shaded}
\begin{Highlighting}[]
\NormalTok{x }\OperatorTok{=}\NormalTok{ (}\StringTok{\textquotesingle{}This will build a very long long \textquotesingle{}}
     \StringTok{\textquotesingle{}long long long long long long string\textquotesingle{}}\NormalTok{)}
\end{Highlighting}
\end{Shaded}
\end{samepage}

Prefer to break lines at the highest possible syntactic level. If you
must break a line twice, break it at the same syntactic level both
times.

\begin{samepage}
\begin{Shaded}
\begin{Highlighting}[]
\NormalTok{Yes: bridgekeeper.answer(}
\NormalTok{         name}\OperatorTok{=}\StringTok{"Arthur"}\NormalTok{, quest}\OperatorTok{=}\NormalTok{questlib.find(owner}\OperatorTok{=}\StringTok{"Arthur"}\NormalTok{, perilous}\OperatorTok{=}\VariableTok{True}\NormalTok{))}

\NormalTok{     answer }\OperatorTok{=}\NormalTok{ (a\_long\_line().of\_chained\_methods()}
\NormalTok{               .that\_eventually\_provides().an\_answer())}

     \ControlFlowTok{if}\NormalTok{ (}
\NormalTok{         config }\KeywordTok{is} \VariableTok{None}
         \KeywordTok{or} \StringTok{\textquotesingle{}editor.language\textquotesingle{}} \KeywordTok{not} \KeywordTok{in}\NormalTok{ config}
         \KeywordTok{or}\NormalTok{ config[}\StringTok{\textquotesingle{}editor.language\textquotesingle{}}\NormalTok{].use\_spaces }\KeywordTok{is} \VariableTok{False}
\NormalTok{     ):}
\NormalTok{       use\_tabs()}
\end{Highlighting}
\end{Shaded}
\end{samepage}

\begin{samepage}
\begin{Shaded}
\begin{Highlighting}[]
\NormalTok{No: bridgekeeper.answer(name}\OperatorTok{=}\StringTok{"Arthur"}\NormalTok{, quest}\OperatorTok{=}\NormalTok{questlib.find(}
\NormalTok{        owner}\OperatorTok{=}\StringTok{"Arthur"}\NormalTok{, perilous}\OperatorTok{=}\VariableTok{True}\NormalTok{))}

\NormalTok{    answer }\OperatorTok{=}\NormalTok{ a\_long\_line().of\_chained\_methods().that\_eventually\_provides(}
\NormalTok{        ).an\_answer()}

    \ControlFlowTok{if}\NormalTok{ (config }\KeywordTok{is} \VariableTok{None} \KeywordTok{or} \StringTok{\textquotesingle{}editor.language\textquotesingle{}} \KeywordTok{not} \KeywordTok{in}\NormalTok{ config }\KeywordTok{or}\NormalTok{ config[}
        \StringTok{\textquotesingle{}editor.language\textquotesingle{}}\NormalTok{].use\_spaces }\KeywordTok{is} \VariableTok{False}\NormalTok{):}
\NormalTok{      use\_tabs()}
\end{Highlighting}
\end{Shaded}
\end{samepage}

Within comments, put long URLs on their own line if necessary.

\begin{samepage}
\begin{Shaded}
\begin{Highlighting}[]
\NormalTok{Yes:  }\CommentTok{\# See details at}
      \CommentTok{\# http://www.example.com/us/developer/documentation/api/content/v2.0/csv\_file\_name\_extension\_full\_specification.html}
\end{Highlighting}
\end{Shaded}
\end{samepage}

\begin{samepage}
\begin{Shaded}
\begin{Highlighting}[]
\NormalTok{No:  }\CommentTok{\# See details at}
     \CommentTok{\# http://www.example.com/us/developer/documentation/api/content/\textbackslash{}}
     \CommentTok{\# v2.0/csv\_file\_name\_extension\_full\_specification.html}
\end{Highlighting}
\end{Shaded}
\end{samepage}

Make note of the indentation of the elements in the line continuation
examples above; see the \hyperref[s3.4-indentation]{indentation} section
for explanation.

In all other cases where a line exceeds 80 characters, and the
\href{https://github.com/psf/black}{Black} or
\href{https://github.com/google/pyink}{Pyink} auto-formatter does not
help bring the line below the limit, the line is allowed to exceed this
maximum. Authors are encouraged to manually break the line up per the
notes above when it is sensible.

\subsection{Parentheses}

Use parentheses sparingly.

It is fine, though not required, to use parentheses around tuples. Do
not use them in return statements or conditional statements unless using
parentheses for implied line continuation or to indicate a tuple.

\begin{samepage}
\begin{Shaded}
\begin{Highlighting}[]
\NormalTok{Yes: }\ControlFlowTok{if}\NormalTok{ foo:}
\NormalTok{         bar()}
     \ControlFlowTok{while}\NormalTok{ x:}
\NormalTok{         x }\OperatorTok{=}\NormalTok{ bar()}
     \ControlFlowTok{if}\NormalTok{ x }\KeywordTok{and}\NormalTok{ y:}
\NormalTok{         bar()}
     \ControlFlowTok{if} \KeywordTok{not}\NormalTok{ x:}
\NormalTok{         bar()}
     \CommentTok{\# For a 1 item tuple the ()s are more visually obvious than the comma.}
\NormalTok{     onesie }\OperatorTok{=}\NormalTok{ (foo,)}
     \ControlFlowTok{return}\NormalTok{ foo}
     \ControlFlowTok{return}\NormalTok{ spam, beans}
     \ControlFlowTok{return}\NormalTok{ (spam, beans)}
     \ControlFlowTok{for}\NormalTok{ (x, y) }\KeywordTok{in} \BuiltInTok{dict}\NormalTok{.items(): ...}
\end{Highlighting}
\end{Shaded}
\end{samepage}

\begin{samepage}
\begin{Shaded}
\begin{Highlighting}[]
\NormalTok{No:  }\ControlFlowTok{if}\NormalTok{ (x):}
\NormalTok{         bar()}
     \ControlFlowTok{if} \KeywordTok{not}\NormalTok{(x):}
\NormalTok{         bar()}
     \ControlFlowTok{return}\NormalTok{ (foo)}
\end{Highlighting}
\end{Shaded}
\end{samepage}

\subsection{Indentation}

Indent your code blocks with \emph{4 spaces}.

Never use tabs. Implied line continuation should align wrapped elements
vertically (see \hyperref[s3.2-line-length]{line length examples}), or
use a hanging 4-space indent. Closing (round, square or curly) brackets
can be placed at the end of the expression, or on separate lines, but
then should be indented the same as the line with the corresponding
opening bracket.

\begin{samepage}
\begin{Shaded}
\begin{Highlighting}[]
\NormalTok{Yes:   }\CommentTok{\# Aligned with opening delimiter.}
\NormalTok{       foo }\OperatorTok{=}\NormalTok{ long\_function\_name(var\_one, var\_two,}
\NormalTok{                                var\_three, var\_four)}
\NormalTok{       meal }\OperatorTok{=}\NormalTok{ (spam,}
\NormalTok{               beans)}

       \CommentTok{\# Aligned with opening delimiter in a dictionary.}
\NormalTok{       foo }\OperatorTok{=}\NormalTok{ \{}
           \StringTok{\textquotesingle{}long\_dictionary\_key\textquotesingle{}}\NormalTok{: value1 }\OperatorTok{+}
\NormalTok{                                  value2,}
\NormalTok{           ...}
\NormalTok{       \}}

       \CommentTok{\# 4{-}space hanging indent; nothing on first line.}
\NormalTok{       foo }\OperatorTok{=}\NormalTok{ long\_function\_name(}
\NormalTok{           var\_one, var\_two, var\_three,}
\NormalTok{           var\_four)}
\NormalTok{       meal }\OperatorTok{=}\NormalTok{ (}
\NormalTok{           spam,}
\NormalTok{           beans)}

       \CommentTok{\# 4{-}space hanging indent; nothing on first line,}
       \CommentTok{\# closing parenthesis on a new line.}
\NormalTok{       foo }\OperatorTok{=}\NormalTok{ long\_function\_name(}
\NormalTok{           var\_one, var\_two, var\_three,}
\NormalTok{           var\_four}
\NormalTok{       )}
\NormalTok{       meal }\OperatorTok{=}\NormalTok{ (}
\NormalTok{           spam,}
\NormalTok{           beans,}
\NormalTok{       )}

       \CommentTok{\# 4{-}space hanging indent in a dictionary.}
\NormalTok{       foo }\OperatorTok{=}\NormalTok{ \{}
           \StringTok{\textquotesingle{}long\_dictionary\_key\textquotesingle{}}\NormalTok{:}
\NormalTok{               long\_dictionary\_value,}
\NormalTok{           ...}
\NormalTok{       \}}
\end{Highlighting}
\end{Shaded}
\end{samepage}

\begin{samepage}
\begin{Shaded}
\begin{Highlighting}[]
\NormalTok{No:    }\CommentTok{\# Stuff on first line forbidden.}
\NormalTok{       foo }\OperatorTok{=}\NormalTok{ long\_function\_name(var\_one, var\_two,}
\NormalTok{           var\_three, var\_four)}
\NormalTok{       meal }\OperatorTok{=}\NormalTok{ (spam,}
\NormalTok{           beans)}

       \CommentTok{\# 2{-}space hanging indent forbidden.}
\NormalTok{       foo }\OperatorTok{=}\NormalTok{ long\_function\_name(}
\NormalTok{         var\_one, var\_two, var\_three,}
\NormalTok{         var\_four)}

       \CommentTok{\# No hanging indent in a dictionary.}
\NormalTok{       foo }\OperatorTok{=}\NormalTok{ \{}
           \StringTok{\textquotesingle{}long\_dictionary\_key\textquotesingle{}}\NormalTok{:}
\NormalTok{           long\_dictionary\_value,}
\NormalTok{           ...}
\NormalTok{       \}}
\end{Highlighting}
\end{Shaded}
\end{samepage}

\paragraph{Trailing commas in sequences of items?}

Trailing commas in sequences of items are recommended only when the
closing container token \texttt{{]}}, \texttt{)}, or \texttt{\}} does
not appear on the same line as the final element, as well as for tuples
with a single element. The presence of a trailing comma is also used as
a hint to our Python code auto-formatter
\href{https://github.com/psf/black}{Black} or
\href{https://github.com/google/pyink}{Pyink} to direct it to
auto-format the container of items to one item per line when the
\texttt{,} after the final element is present.

\begin{samepage}
\begin{Shaded}
\begin{Highlighting}[]
\NormalTok{Yes:   golomb3 }\OperatorTok{=}\NormalTok{ [}\DecValTok{0}\NormalTok{, }\DecValTok{1}\NormalTok{, }\DecValTok{3}\NormalTok{]}
\NormalTok{       golomb4 }\OperatorTok{=}\NormalTok{ [}
           \DecValTok{0}\NormalTok{,}
           \DecValTok{1}\NormalTok{,}
           \DecValTok{4}\NormalTok{,}
           \DecValTok{6}\NormalTok{,}
\NormalTok{       ]}
\end{Highlighting}
\end{Shaded}
\end{samepage}

\begin{samepage}
\begin{Shaded}
\begin{Highlighting}[]
\NormalTok{No:    golomb4 }\OperatorTok{=}\NormalTok{ [}
           \DecValTok{0}\NormalTok{,}
           \DecValTok{1}\NormalTok{,}
           \DecValTok{4}\NormalTok{,}
           \DecValTok{6}\NormalTok{,]}
\end{Highlighting}
\end{Shaded}
\end{samepage}

\subsection{Blank Lines}

Two blank lines between top-level definitions, be they function or class
definitions. One blank line between method definitions and between the
docstring of a \texttt{class} and the first method. No blank line
following a \texttt{def} line. Use single blank lines as you judge
appropriate within functions or methods.

Blank lines need not be anchored to the definition. For example, related
comments immediately preceding function, class, and method definitions
can make sense. Consider if your comment might be more useful as part of
the docstring.

\subsection{Whitespace}

Follow standard typographic rules for the use of spaces around
punctuation.

No whitespace inside parentheses, brackets or braces.

\begin{samepage}
\begin{Shaded}
\begin{Highlighting}[]
\NormalTok{Yes: spam(ham[}\DecValTok{1}\NormalTok{], \{}\StringTok{\textquotesingle{}eggs\textquotesingle{}}\NormalTok{: }\DecValTok{2}\NormalTok{\}, [])}
\end{Highlighting}
\end{Shaded}
\end{samepage}

\begin{samepage}
\begin{Shaded}
\begin{Highlighting}[]
\NormalTok{No:  spam( ham[ }\DecValTok{1}\NormalTok{ ], \{ }\StringTok{\textquotesingle{}eggs\textquotesingle{}}\NormalTok{: }\DecValTok{2}\NormalTok{ \}, [ ] )}
\end{Highlighting}
\end{Shaded}
\end{samepage}

No whitespace before a comma, semicolon, or colon. Do use whitespace
after a comma, semicolon, or colon, except at the end of the line.

\begin{samepage}
\begin{Shaded}
\begin{Highlighting}[]
\NormalTok{Yes: }\ControlFlowTok{if}\NormalTok{ x }\OperatorTok{==} \DecValTok{4}\NormalTok{:}
         \BuiltInTok{print}\NormalTok{(x, y)}
\NormalTok{     x, y }\OperatorTok{=}\NormalTok{ y, x}
\end{Highlighting}
\end{Shaded}
\end{samepage}

\begin{samepage}
\begin{Shaded}
\begin{Highlighting}[]
\NormalTok{No:  }\ControlFlowTok{if}\NormalTok{ x }\OperatorTok{==} \DecValTok{4}\NormalTok{ :}
         \BuiltInTok{print}\NormalTok{(x , y)}
\NormalTok{     x , y }\OperatorTok{=}\NormalTok{ y , x}
\end{Highlighting}
\end{Shaded}
\end{samepage}

No whitespace before the open paren/bracket that starts an argument
list, indexing or slicing.

\begin{samepage}
\begin{Shaded}
\begin{Highlighting}[]
\NormalTok{Yes: spam(}\DecValTok{1}\NormalTok{)}
\end{Highlighting}
\end{Shaded}
\end{samepage}

\begin{samepage}
\begin{Shaded}
\begin{Highlighting}[]
\NormalTok{No:  spam (}\DecValTok{1}\NormalTok{)}
\end{Highlighting}
\end{Shaded}
\end{samepage}

\begin{samepage}
\begin{Shaded}
\begin{Highlighting}[]
\NormalTok{Yes: }\BuiltInTok{dict}\NormalTok{[}\StringTok{\textquotesingle{}key\textquotesingle{}}\NormalTok{] }\OperatorTok{=} \BuiltInTok{list}\NormalTok{[index]}
\end{Highlighting}
\end{Shaded}
\end{samepage}

\begin{samepage}
\begin{Shaded}
\begin{Highlighting}[]
\NormalTok{No:  }\BuiltInTok{dict}\NormalTok{ [}\StringTok{\textquotesingle{}key\textquotesingle{}}\NormalTok{] }\OperatorTok{=} \BuiltInTok{list}\NormalTok{ [index]}
\end{Highlighting}
\end{Shaded}
\end{samepage}

No trailing whitespace.

Surround binary operators with a single space on either side for
assignment (\texttt{=}), comparisons
(\texttt{==,\ \textless{},\ \textgreater{},\ !=,\ \textless{}\textgreater{},\ \textless{}=,\ \textgreater{}=,\ in,\ not\ in,\ is,\ is\ not}),
and Booleans (\texttt{and,\ or,\ not}). Use your better judgment for the
insertion of spaces around arithmetic operators (\texttt{+}, \texttt{-},
\texttt{*}, \texttt{/}, \texttt{//}, \texttt{\%}, \texttt{**},
\texttt{@}).

\begin{samepage}
\begin{Shaded}
\begin{Highlighting}[]
\NormalTok{Yes: x }\OperatorTok{==} \DecValTok{1}
\end{Highlighting}
\end{Shaded}
\end{samepage}

\begin{samepage}
\begin{Shaded}
\begin{Highlighting}[]
\NormalTok{No:  x}\OperatorTok{\textless{}}\DecValTok{1}
\end{Highlighting}
\end{Shaded}
\end{samepage}

Never use spaces around \texttt{=} when passing keyword arguments or
defining a default parameter value, with one exception:
\hyperref[typing-default-values]{when a type annotation is present},
\emph{do} use spaces around the \texttt{=} for the default parameter
value.

\begin{samepage}
\begin{Shaded}
\begin{Highlighting}[]
\NormalTok{Yes: }\KeywordTok{def} \BuiltInTok{complex}\NormalTok{(real, imag}\OperatorTok{=}\FloatTok{0.0}\NormalTok{): }\ControlFlowTok{return}\NormalTok{ Magic(r}\OperatorTok{=}\NormalTok{real, i}\OperatorTok{=}\NormalTok{imag)}
\NormalTok{Yes: }\KeywordTok{def} \BuiltInTok{complex}\NormalTok{(real, imag: }\BuiltInTok{float} \OperatorTok{=} \FloatTok{0.0}\NormalTok{): }\ControlFlowTok{return}\NormalTok{ Magic(r}\OperatorTok{=}\NormalTok{real, i}\OperatorTok{=}\NormalTok{imag)}
\end{Highlighting}
\end{Shaded}
\end{samepage}

\begin{samepage}
\begin{Shaded}
\begin{Highlighting}[]
\NormalTok{No:  }\KeywordTok{def} \BuiltInTok{complex}\NormalTok{(real, imag }\OperatorTok{=} \FloatTok{0.0}\NormalTok{): }\ControlFlowTok{return}\NormalTok{ Magic(r }\OperatorTok{=}\NormalTok{ real, i }\OperatorTok{=}\NormalTok{ imag)}
\NormalTok{No:  }\KeywordTok{def} \BuiltInTok{complex}\NormalTok{(real, imag: }\BuiltInTok{float}\OperatorTok{=}\FloatTok{0.0}\NormalTok{): }\ControlFlowTok{return}\NormalTok{ Magic(r }\OperatorTok{=}\NormalTok{ real, i }\OperatorTok{=}\NormalTok{ imag)}
\end{Highlighting}
\end{Shaded}
\end{samepage}

Don't use spaces to vertically align tokens on consecutive lines, since
it becomes a maintenance burden (applies to \texttt{:}, \texttt{\#},
\texttt{=}, etc.):

\begin{samepage}
\begin{Shaded}
\begin{Highlighting}[]
\NormalTok{Yes:}
\NormalTok{  foo }\OperatorTok{=} \DecValTok{1000}  \CommentTok{\# comment}
\NormalTok{  long\_name }\OperatorTok{=} \DecValTok{2}  \CommentTok{\# comment that should not be aligned}

\NormalTok{  dictionary }\OperatorTok{=}\NormalTok{ \{}
      \StringTok{\textquotesingle{}foo\textquotesingle{}}\NormalTok{: }\DecValTok{1}\NormalTok{,}
      \StringTok{\textquotesingle{}long\_name\textquotesingle{}}\NormalTok{: }\DecValTok{2}\NormalTok{,}
\NormalTok{  \}}
\end{Highlighting}
\end{Shaded}
\end{samepage}

\begin{samepage}
\begin{Shaded}
\begin{Highlighting}[]
\NormalTok{No:}
\NormalTok{  foo       }\OperatorTok{=} \DecValTok{1000}  \CommentTok{\# comment}
\NormalTok{  long\_name }\OperatorTok{=} \DecValTok{2}     \CommentTok{\# comment that should not be aligned}

\NormalTok{  dictionary }\OperatorTok{=}\NormalTok{ \{}
      \StringTok{\textquotesingle{}foo\textquotesingle{}}\NormalTok{      : }\DecValTok{1}\NormalTok{,}
      \StringTok{\textquotesingle{}long\_name\textquotesingle{}}\NormalTok{: }\DecValTok{2}\NormalTok{,}
\NormalTok{  \}}
\end{Highlighting}
\end{Shaded}
\end{samepage}

\subsection{Shebang Line}

Most \texttt{.py} files do not need to start with a \texttt{\#!} line.
Start the main file of a program with \texttt{\#!/usr/bin/env\ python3}
(to support virtualenvs) or \texttt{\#!/usr/bin/python3} per
\href{https://peps.python.org/pep-0394/}{PEP-394}.

This line is used by the kernel to find the Python interpreter, but is
ignored by Python when importing modules. It is only necessary on a file
intended to be executed directly.

\subsection{Comments and Docstrings}

Be sure to use the right style for module, function, method docstrings
and inline comments.

\paragraph{Docstrings}

Python uses \emph{docstrings} to document code. A docstring is a string
that is the first statement in a package, module, class or function.
These strings can be extracted automatically through the
\texttt{\_\_doc\_\_} member of the object and are used by
\texttt{pydoc}. (Try running \texttt{pydoc} on your module to see how it
looks.) Always use the three-double-quote \texttt{"""} format for
docstrings (per \href{https://peps.python.org/pep-0257/}{PEP 257}). A
docstring should be organized as a summary line (one physical line not
exceeding 80 characters) terminated by a period, question mark, or
exclamation point. When writing more (encouraged), this must be followed
by a blank line, followed by the rest of the docstring starting at the
same cursor position as the first quote of the first line. There are
more formatting guidelines for docstrings below.

\paragraph{Modules}

Every file should contain license boilerplate. Choose the appropriate
boilerplate for the license used by the project (for example, Apache
2.0, BSD, LGPL, GPL).

Files should start with a docstring describing the contents and usage of
the module.

\begin{samepage}
\begin{Shaded}
\begin{Highlighting}[]
\CommentTok{"""A one{-}line summary of the module or program, terminated by a period.}

\CommentTok{Leave one blank line.  The rest of this docstring should contain an}
\CommentTok{overall description of the module or program.  Optionally, it may also}
\CommentTok{contain a brief description of exported classes and functions and/or usage}
\CommentTok{examples.}

\CommentTok{Typical usage example:}

\CommentTok{  foo = ClassFoo()}
\CommentTok{  bar = foo.FunctionBar()}
\CommentTok{"""}
\end{Highlighting}
\end{Shaded}
\end{samepage}

\subparagraph{Test modules}

Module-level docstrings for test files are not required. They should be
included only when there is additional information that can be provided.

Examples include some specifics on how the test should be run, an
explanation of an unusual setup pattern, dependency on the external
environment, and so on.

\begin{samepage}
\begin{Shaded}
\begin{Highlighting}[]
\CommentTok{"""This blaze test uses golden files.}

\CommentTok{You can update those files by running}
\CommentTok{\textasciigrave{}blaze run //foo/bar:foo\_test {-}{-} {-}{-}update\_golden\_files\textasciigrave{} from the \textasciigrave{}google3\textasciigrave{}}
\CommentTok{directory.}
\CommentTok{"""}
\end{Highlighting}
\end{Shaded}
\end{samepage}

Docstrings that do not provide any new information should not be used.

\begin{samepage}
\begin{Shaded}
\begin{Highlighting}[]
\CommentTok{"""Tests for foo.bar."""}
\end{Highlighting}
\end{Shaded}
\end{samepage}

\paragraph{Functions and Methods}

In this section, ``function'' means a method, function, generator, or
property.

A docstring is mandatory for every function that has one or more of the
following properties:

\begin{itemize}
\tightlist
\item
  being part of the public API
\item
  nontrivial size
\item
  non-obvious logic
\end{itemize}

A docstring should give enough information to write a call to the
function without reading the function's code. The docstring should
describe the function's calling syntax and its semantics, but generally
not its implementation details, unless those details are relevant to how
the function is to be used. For example, a function that mutates one of
its arguments as a side effect should note that in its docstring.
Otherwise, subtle but important details of a function's implementation
that are not relevant to the caller are better expressed as comments
alongside the code than within the function's docstring.

The docstring may be descriptive-style
(\texttt{"""Fetches\ rows\ from\ a\ Bigtable."""}) or imperative-style
(\texttt{"""Fetch\ rows\ from\ a\ Bigtable."""}), but the style should
be consistent within a file. The docstring for a \texttt{@property} data
descriptor should use the same style as the docstring for an attribute
or a function argument (\texttt{"""The\ Bigtable\ path."""}, rather than
\texttt{"""Returns\ the\ Bigtable\ path."""}).

Certain aspects of a function should be documented in special sections,
listed below. Each section begins with a heading line, which ends with a
colon. All sections other than the heading should maintain a hanging
indent of two or four spaces (be consistent within a file). These
sections can be omitted in cases where the function's name and signature
are informative enough that it can be aptly described using a one-line
docstring.

\hyperref[doc-function-args]{\emph{Args:}} : List each parameter by
name. A description should follow the name, and be separated by a colon
followed by either a space or newline. If the description is too long to
fit on a single 80-character line, use a hanging indent of 2 or 4 spaces
more than the parameter name (be consistent with the rest of the
docstrings in the file). The description should include required type(s)
if the code does not contain a corresponding type annotation. If a
function accepts \texttt{*foo} (variable length argument lists) and/or
\texttt{**bar} (arbitrary keyword arguments), they should be listed as
\texttt{*foo} and \texttt{**bar}.

\hyperref[doc-function-returns]{\emph{Returns:} (or \emph{Yields:} for
generators)} : Describe the semantics of the return value, including any
type information that the type annotation does not provide. If the
function only returns None, this section is not required. It may also be
omitted if the docstring starts with ``Return'', ``Returns'', ``Yield'',
or ``Yields''
(e.g.~\texttt{"""Returns\ row\ \ \ \ \ from\ Bigtable\ as\ a\ tuple\ of\ strings."""})
\emph{and} the opening sentence is sufficient to describe the return
value. Do not imitate older `NumPy style'
(\href{https://numpy.org/doc/1.24/reference/generated/numpy.linalg.qr.html}{example}),
which frequently documented a tuple return value as if it were multiple
return values with individual names (never mentioning the tuple).
Instead, describe such a return value as: ``Returns: A tuple (mat\_a,
mat\_b), where mat\_a is \ldots, and \ldots{}''. The auxiliary names in
the docstring need not necessarily correspond to any internal names used
in the function body (as those are not part of the API). If the function
uses \texttt{yield} (is a generator), the \texttt{Yields:} section
should document the object returned by \texttt{next()}, instead of the
generator object itself that the call evaluates to.

\hyperref[doc-function-raises]{\emph{Raises:}} : List all exceptions
that are relevant to the interface followed by a description. Use a
similar exception name + colon + space or newline and hanging indent
style as described in \emph{Args:}. You should not document exceptions
that get raised if the API specified in the docstring is violated
(because this would paradoxically make behavior under violation of the
API part of the API).

\begin{samepage}
\begin{Shaded}
\begin{Highlighting}[]
\KeywordTok{def}\NormalTok{ fetch\_smalltable\_rows(}
\NormalTok{    table\_handle: smalltable.Table,}
\NormalTok{    keys: Sequence[}\BuiltInTok{bytes} \OperatorTok{|} \BuiltInTok{str}\NormalTok{],}
\NormalTok{    require\_all\_keys: }\BuiltInTok{bool} \OperatorTok{=} \VariableTok{False}\NormalTok{,}
\NormalTok{) }\OperatorTok{{-}\textgreater{}}\NormalTok{ Mapping[}\BuiltInTok{bytes}\NormalTok{, }\BuiltInTok{tuple}\NormalTok{[}\BuiltInTok{str}\NormalTok{, ...]]:}
    \CommentTok{"""Fetches rows from a Smalltable.}

\CommentTok{    Retrieves rows pertaining to the given keys from the Table instance}
\CommentTok{    represented by table\_handle.  String keys will be UTF{-}8 encoded.}

\CommentTok{    Args:}
\CommentTok{        table\_handle: An open smalltable.Table instance.}
\CommentTok{        keys: A sequence of strings representing the key of each table}
\CommentTok{          row to fetch.  String keys will be UTF{-}8 encoded.}
\CommentTok{        require\_all\_keys: If True only rows with values set for all keys will be}
\CommentTok{          returned.}

\CommentTok{    Returns:}
\CommentTok{        A dict mapping keys to the corresponding table row data}
\CommentTok{        fetched. Each row is represented as a tuple of strings. For}
\CommentTok{        example:}

\CommentTok{        \{b\textquotesingle{}Serak\textquotesingle{}: (\textquotesingle{}Rigel VII\textquotesingle{}, \textquotesingle{}Preparer\textquotesingle{}),}
\CommentTok{         b\textquotesingle{}Zim\textquotesingle{}: (\textquotesingle{}Irk\textquotesingle{}, \textquotesingle{}Invader\textquotesingle{}),}
\CommentTok{         b\textquotesingle{}Lrrr\textquotesingle{}: (\textquotesingle{}Omicron Persei 8\textquotesingle{}, \textquotesingle{}Emperor\textquotesingle{})\}}

\CommentTok{        Returned keys are always bytes.  If a key from the keys argument is}
\CommentTok{        missing from the dictionary, then that row was not found in the}
\CommentTok{        table (and require\_all\_keys must have been False).}

\CommentTok{    Raises:}
\CommentTok{        IOError: An error occurred accessing the smalltable.}
\CommentTok{    """}
\end{Highlighting}
\end{Shaded}
\end{samepage}

\clearpage
Similarly, this variation on \texttt{Args:} with a line break is also
allowed:

\begin{samepage}
\begin{Shaded}
\begin{Highlighting}[]
\KeywordTok{def}\NormalTok{ fetch\_smalltable\_rows(}
\NormalTok{    table\_handle: smalltable.Table,}
\NormalTok{    keys: Sequence[}\BuiltInTok{bytes} \OperatorTok{|} \BuiltInTok{str}\NormalTok{],}
\NormalTok{    require\_all\_keys: }\BuiltInTok{bool} \OperatorTok{=} \VariableTok{False}\NormalTok{,}
\NormalTok{) }\OperatorTok{{-}\textgreater{}}\NormalTok{ Mapping[}\BuiltInTok{bytes}\NormalTok{, }\BuiltInTok{tuple}\NormalTok{[}\BuiltInTok{str}\NormalTok{, ...]]:}
    \CommentTok{"""Fetches rows from a Smalltable.}

\CommentTok{    Retrieves rows pertaining to the given keys from the Table instance}
\CommentTok{    represented by table\_handle.  String keys will be UTF{-}8 encoded.}

\CommentTok{    Args:}
\CommentTok{      table\_handle:}
\CommentTok{        An open smalltable.Table instance.}
\CommentTok{      keys:}
\CommentTok{        A sequence of strings representing the key of each table row to}
\CommentTok{        fetch.  String keys will be UTF{-}8 encoded.}
\CommentTok{      require\_all\_keys:}
\CommentTok{        If True only rows with values set for all keys will be returned.}

\CommentTok{    Returns:}
\CommentTok{      A dict mapping keys to the corresponding table row data}
\CommentTok{      fetched. Each row is represented as a tuple of strings. For}
\CommentTok{      example:}

\CommentTok{      \{b\textquotesingle{}Serak\textquotesingle{}: (\textquotesingle{}Rigel VII\textquotesingle{}, \textquotesingle{}Preparer\textquotesingle{}),}
\CommentTok{       b\textquotesingle{}Zim\textquotesingle{}: (\textquotesingle{}Irk\textquotesingle{}, \textquotesingle{}Invader\textquotesingle{}),}
\CommentTok{       b\textquotesingle{}Lrrr\textquotesingle{}: (\textquotesingle{}Omicron Persei 8\textquotesingle{}, \textquotesingle{}Emperor\textquotesingle{})\}}

\CommentTok{      Returned keys are always bytes.  If a key from the keys argument is}
\CommentTok{      missing from the dictionary, then that row was not found in the}
\CommentTok{      table (and require\_all\_keys must have been False).}

\CommentTok{    Raises:}
\CommentTok{      IOError: An error occurred accessing the smalltable.}
\CommentTok{    """}
\end{Highlighting}
\end{Shaded}
\end{samepage}

\subparagraph{Overridden Methods}

A method that overrides a method from a base class does not need a
docstring if it is explicitly decorated with
\href{https://typing-extensions.readthedocs.io/en/latest/\#override}{\texttt{@override}}
(from \texttt{typing\_extensions} or \texttt{typing} modules), unless
the overriding method's behavior materially refines the base method's
contract, or details need to be provided (e.g., documenting additional
side effects), in which case a docstring with at least those differences
is required on the overriding method.

\begin{samepage}
\begin{Shaded}
\begin{Highlighting}[]
\ImportTok{from}\NormalTok{ typing\_extensions }\ImportTok{import}\NormalTok{ override}

\KeywordTok{class}\NormalTok{ Parent:}
  \KeywordTok{def}\NormalTok{ do\_something(}\VariableTok{self}\NormalTok{):}
    \CommentTok{"""Parent method, includes docstring."""}

\CommentTok{\# Child class, method annotated with override.}
\KeywordTok{class}\NormalTok{ Child(Parent):}
  \AttributeTok{@override}
  \KeywordTok{def}\NormalTok{ do\_something(}\VariableTok{self}\NormalTok{):}
    \ControlFlowTok{pass}
\end{Highlighting}
\end{Shaded}
\end{samepage}

\begin{samepage}
\begin{Shaded}
\begin{Highlighting}[]
\CommentTok{\# Child class, but without @override decorator, a docstring is required.}
\KeywordTok{class}\NormalTok{ Child(Parent):}
  \KeywordTok{def}\NormalTok{ do\_something(}\VariableTok{self}\NormalTok{):}
    \ControlFlowTok{pass}

\CommentTok{\# Docstring is trivial, @override is sufficient to indicate that docs can be}
\CommentTok{\# found in the base class.}
\KeywordTok{class}\NormalTok{ Child(Parent):}
  \AttributeTok{@override}
  \KeywordTok{def}\NormalTok{ do\_something(}\VariableTok{self}\NormalTok{):}
    \CommentTok{"""See base class."""}
\end{Highlighting}
\end{Shaded}
\end{samepage}

\paragraph{Classes}

Classes should have a docstring below the class definition describing
the class. Public attributes, excluding
\hyperref[properties]{properties}, should be documented here in an
\texttt{Attributes} section and follow the same formatting as a
\hyperref[doc-function-args]{function's \texttt{Args}} section.

\begin{samepage}
\begin{Shaded}
\begin{Highlighting}[]
\KeywordTok{class}\NormalTok{ SampleClass:}
    \CommentTok{"""Summary of class here.}

\CommentTok{    Longer class information...}
\CommentTok{    Longer class information...}

\CommentTok{    Attributes:}
\CommentTok{        likes\_spam: A boolean indicating if we like SPAM or not.}
\CommentTok{        eggs: An integer count of the eggs we have laid.}
\CommentTok{    """}

    \KeywordTok{def} \FunctionTok{\_\_init\_\_}\NormalTok{(}\VariableTok{self}\NormalTok{, likes\_spam: }\BuiltInTok{bool} \OperatorTok{=} \VariableTok{False}\NormalTok{):}
        \CommentTok{"""Initializes the instance based on spam preference.}

\CommentTok{        Args:}
\CommentTok{          likes\_spam: Defines if instance exhibits this preference.}
\CommentTok{        """}
        \VariableTok{self}\NormalTok{.likes\_spam }\OperatorTok{=}\NormalTok{ likes\_spam}
        \VariableTok{self}\NormalTok{.eggs }\OperatorTok{=} \DecValTok{0}

    \AttributeTok{@property}
    \KeywordTok{def}\NormalTok{ butter\_sticks(}\VariableTok{self}\NormalTok{) }\OperatorTok{{-}\textgreater{}} \BuiltInTok{int}\NormalTok{:}
        \CommentTok{"""The number of butter sticks we have."""}
\end{Highlighting}
\end{Shaded}
\end{samepage}

All class docstrings should start with a one-line summary that describes
what the class instance represents. This implies that subclasses of
\texttt{Exception} should also describe what the exception represents,
and not the context in which it might occur. The class docstring should
not repeat unnecessary information, such as that the class is a class.

\begin{samepage}
\begin{Shaded}
\begin{Highlighting}[]
\CommentTok{\# Yes:}
\KeywordTok{class}\NormalTok{ CheeseShopAddress:}
  \CommentTok{"""The address of a cheese shop.}

\CommentTok{  ...}
\CommentTok{  """}

\KeywordTok{class}\NormalTok{ OutOfCheeseError(}\PreprocessorTok{Exception}\NormalTok{):}
  \CommentTok{"""No more cheese is available."""}
\end{Highlighting}
\end{Shaded}
\end{samepage}

\begin{samepage}
\begin{Shaded}
\begin{Highlighting}[]
\CommentTok{\# No:}
\KeywordTok{class}\NormalTok{ CheeseShopAddress:}
  \CommentTok{"""Class that describes the address of a cheese shop.}

\CommentTok{  ...}
\CommentTok{  """}

\KeywordTok{class}\NormalTok{ OutOfCheeseError(}\PreprocessorTok{Exception}\NormalTok{):}
  \CommentTok{"""Raised when no more cheese is available."""}
\end{Highlighting}
\end{Shaded}
\end{samepage}

\paragraph{Block and Inline Comments}

The final place to have comments is in tricky parts of the code. If
you're going to have to explain it at the next
\href{http://en.wikipedia.org/wiki/Code_review}{code review}, you should
comment it now. Complicated operations get a few lines of comments
before the operations commence. Non-obvious ones get comments at the end
of the line.

\begin{samepage}
\begin{Shaded}
\begin{Highlighting}[]
\CommentTok{\# We use a weighted dictionary search to find out where i is in}
\CommentTok{\# the array.  We extrapolate position based on the largest num}
\CommentTok{\# in the array and the array size and then do binary search to}
\CommentTok{\# get the exact number.}

\ControlFlowTok{if}\NormalTok{ i }\OperatorTok{\&}\NormalTok{ (i}\OperatorTok{{-}}\DecValTok{1}\NormalTok{) }\OperatorTok{==} \DecValTok{0}\NormalTok{:  }\CommentTok{\# True if i is 0 or a power of 2.}
\end{Highlighting}
\end{Shaded}
\end{samepage}

To improve legibility, these comments should start at least 2 spaces
away from the code with the comment character \texttt{\#}, followed by
at least one space before the text of the comment itself.

On the other hand, never describe the code. Assume the person reading
the code knows Python (though not what you're trying to do) better than
you do.

\begin{samepage}
\begin{Shaded}
\begin{Highlighting}[]
\CommentTok{\# BAD COMMENT: Now go through the b array and make sure whenever i occurs}
\CommentTok{\# the next element is i+1}
\end{Highlighting}
\end{Shaded}
\end{samepage}

\paragraph{Punctuation, Spelling, and Grammar}

Pay attention to punctuation, spelling, and grammar; it is easier to
read well-written comments than badly written ones.

Comments should be as readable as narrative text, with proper
capitalization and punctuation. In many cases, complete sentences are
more readable than sentence fragments. Shorter comments, such as
comments at the end of a line of code, can sometimes be less formal, but
you should be consistent with your style.

Although it can be frustrating to have a code reviewer point out that
you are using a comma when you should be using a semicolon, it is very
important that source code maintain a high level of clarity and
readability. Proper punctuation, spelling, and grammar help with that
goal.

\subsection{Strings}

Use an
\href{https://docs.python.org/3/reference/lexical_analysis.html\#f-strings}{f-string},
the \texttt{\%} operator, or the \texttt{format} method for formatting
strings, even when the parameters are all strings. Use your best
judgment to decide between string formatting options. A single join with
\texttt{+} is okay but do not format with \texttt{+}.

\begin{samepage}
\begin{Shaded}
\begin{Highlighting}[]
\NormalTok{Yes: x }\OperatorTok{=} \SpecialStringTok{f\textquotesingle{}name: }\SpecialCharTok{\{}\NormalTok{name}\SpecialCharTok{\}}\SpecialStringTok{; score: }\SpecialCharTok{\{}\NormalTok{n}\SpecialCharTok{\}}\SpecialStringTok{\textquotesingle{}}
\NormalTok{     x }\OperatorTok{=} \StringTok{\textquotesingle{}}\SpecialCharTok{\%s}\StringTok{, }\SpecialCharTok{\%s}\StringTok{!\textquotesingle{}} \OperatorTok{\%}\NormalTok{ (imperative, expletive)}
\NormalTok{     x }\OperatorTok{=} \StringTok{\textquotesingle{}}\SpecialCharTok{\{\}}\StringTok{, }\SpecialCharTok{\{\}}\StringTok{\textquotesingle{}}\NormalTok{.}\BuiltInTok{format}\NormalTok{(first, second)}
\NormalTok{     x }\OperatorTok{=} \StringTok{\textquotesingle{}name: }\SpecialCharTok{\%s}\StringTok{; score: }\SpecialCharTok{\%d}\StringTok{\textquotesingle{}} \OperatorTok{\%}\NormalTok{ (name, n)}
\NormalTok{     x }\OperatorTok{=} \StringTok{\textquotesingle{}name: }\SpecialCharTok{\%(name)s}\StringTok{; score: }\SpecialCharTok{\%(score)d}\StringTok{\textquotesingle{}} \OperatorTok{\%}\NormalTok{ \{}\StringTok{\textquotesingle{}name\textquotesingle{}}\NormalTok{:name, }\StringTok{\textquotesingle{}score\textquotesingle{}}\NormalTok{:n\}}
\NormalTok{     x }\OperatorTok{=} \StringTok{\textquotesingle{}name: }\SpecialCharTok{\{\}}\StringTok{; score: }\SpecialCharTok{\{\}}\StringTok{\textquotesingle{}}\NormalTok{.}\BuiltInTok{format}\NormalTok{(name, n)}
\NormalTok{     x }\OperatorTok{=}\NormalTok{ a }\OperatorTok{+}\NormalTok{ b}
\end{Highlighting}
\end{Shaded}
\end{samepage}

\begin{samepage}
\begin{Shaded}
\begin{Highlighting}[]
\NormalTok{No: x }\OperatorTok{=}\NormalTok{ first }\OperatorTok{+} \StringTok{\textquotesingle{}, \textquotesingle{}} \OperatorTok{+}\NormalTok{ second}
\NormalTok{    x }\OperatorTok{=} \StringTok{\textquotesingle{}name: \textquotesingle{}} \OperatorTok{+}\NormalTok{ name }\OperatorTok{+} \StringTok{\textquotesingle{}; score: \textquotesingle{}} \OperatorTok{+} \BuiltInTok{str}\NormalTok{(n)}
\end{Highlighting}
\end{Shaded}
\end{samepage}

Avoid using the \texttt{+} and \texttt{+=} operators to accumulate a
string within a loop. In some conditions, accumulating a string with
addition can lead to quadratic rather than linear running time. Although
common accumulations of this sort may be optimized on CPython, that is
an implementation detail. The conditions under which an optimization
applies are not easy to predict and may change. Instead, add each
substring to a list and
\texttt{\textquotesingle{}\textquotesingle{}.join} the list after the
loop terminates, or write each substring to an \texttt{io.StringIO}
buffer. These techniques consistently have amortized-linear run-time
complexity.

\begin{samepage}
\begin{Shaded}
\begin{Highlighting}[]
\NormalTok{Yes: items }\OperatorTok{=}\NormalTok{ [}\StringTok{\textquotesingle{}\textless{}table\textgreater{}\textquotesingle{}}\NormalTok{]}
     \ControlFlowTok{for}\NormalTok{ last\_name, first\_name }\KeywordTok{in}\NormalTok{ employee\_list:}
\NormalTok{         items.append(}\StringTok{\textquotesingle{}\textless{}tr\textgreater{}\textless{}td\textgreater{}}\SpecialCharTok{\%s}\StringTok{, }\SpecialCharTok{\%s}\StringTok{\textless{}/td\textgreater{}\textless{}/tr\textgreater{}\textquotesingle{}} \OperatorTok{\%}\NormalTok{ (last\_name, first\_name))}
\NormalTok{     items.append(}\StringTok{\textquotesingle{}\textless{}/table\textgreater{}\textquotesingle{}}\NormalTok{)}
\NormalTok{     employee\_table }\OperatorTok{=} \StringTok{\textquotesingle{}\textquotesingle{}}\NormalTok{.join(items)}
\end{Highlighting}
\end{Shaded}
\end{samepage}

\begin{samepage}
\begin{Shaded}
\begin{Highlighting}[]
\NormalTok{No: employee\_table }\OperatorTok{=} \StringTok{\textquotesingle{}\textless{}table\textgreater{}\textquotesingle{}}
    \ControlFlowTok{for}\NormalTok{ last\_name, first\_name }\KeywordTok{in}\NormalTok{ employee\_list:}
\NormalTok{        employee\_table }\OperatorTok{+=} \StringTok{\textquotesingle{}\textless{}tr\textgreater{}\textless{}td\textgreater{}}\SpecialCharTok{\%s}\StringTok{, }\SpecialCharTok{\%s}\StringTok{\textless{}/td\textgreater{}\textless{}/tr\textgreater{}\textquotesingle{}} \OperatorTok{\%}\NormalTok{ (last\_name, first\_name)}
\NormalTok{    employee\_table }\OperatorTok{+=} \StringTok{\textquotesingle{}\textless{}/table\textgreater{}\textquotesingle{}}
\end{Highlighting}
\end{Shaded}
\end{samepage}

Be consistent with your choice of string quote character within a file.
Pick \texttt{\textquotesingle{}} or \texttt{"} and stick with it. It is
okay to use the other quote character on a string to avoid the need to
backslash-escape quote characters within the string.

\begin{samepage}
\begin{Shaded}
\begin{Highlighting}[]
\NormalTok{Yes:}
\NormalTok{  Python(}\StringTok{\textquotesingle{}Why are you hiding your eyes?\textquotesingle{}}\NormalTok{)}
\NormalTok{  Gollum(}\StringTok{"I\textquotesingle{}m scared of lint errors."}\NormalTok{)}
\NormalTok{  Narrator(}\StringTok{\textquotesingle{}"Good!" thought a happy Python reviewer.\textquotesingle{}}\NormalTok{)}
\end{Highlighting}
\end{Shaded}
\end{samepage}

\begin{samepage}
\begin{Shaded}
\begin{Highlighting}[]
\NormalTok{No:}
\NormalTok{  Python(}\StringTok{"Why are you hiding your eyes?"}\NormalTok{)}
\NormalTok{  Gollum(}\StringTok{\textquotesingle{}The lint. It burns. It burns us.\textquotesingle{}}\NormalTok{)}
\NormalTok{  Gollum(}\StringTok{"Always the great lint. Watching. Watching."}\NormalTok{)}
\end{Highlighting}
\end{Shaded}
\end{samepage}

Prefer \texttt{"""} for multi-line strings rather than
\texttt{\textquotesingle{}\textquotesingle{}\textquotesingle{}}.
Projects may choose to use
\texttt{\textquotesingle{}\textquotesingle{}\textquotesingle{}} for all
non-docstring multi-line strings if and only if they also use
\texttt{\textquotesingle{}} for regular strings. Docstrings must use
\texttt{"""} regardless.

Multi-line strings do not flow with the indentation of the rest of the
program. If you need to avoid embedding extra space in the string, use
either concatenated single-line strings or a multi-line string with
\href{https://docs.python.org/3/library/textwrap.html\#textwrap.dedent}{\texttt{textwrap.dedent()}}
to remove the initial space on each line:

\begin{samepage}
\begin{Shaded}
\begin{Highlighting}[]
\NormalTok{  No:}
\NormalTok{  long\_string }\OperatorTok{=} \StringTok{"""This is pretty ugly.}
\StringTok{Don\textquotesingle{}t do this.}
\StringTok{"""}
\end{Highlighting}
\end{Shaded}
\end{samepage}

\begin{samepage}
\begin{Shaded}
\begin{Highlighting}[]
\NormalTok{  Yes:}
\NormalTok{  long\_string }\OperatorTok{=} \StringTok{"""This is fine if your use case can accept}
\StringTok{      extraneous leading spaces."""}
\end{Highlighting}
\end{Shaded}
\end{samepage}

\begin{samepage}
\begin{Shaded}
\begin{Highlighting}[]
\NormalTok{  Yes:}
\NormalTok{  long\_string }\OperatorTok{=}\NormalTok{ (}\StringTok{"And this is fine if you cannot accept}\CharTok{\textbackslash{}n}\StringTok{"} \OperatorTok{+}
                 \StringTok{"extraneous leading spaces."}\NormalTok{)}
\end{Highlighting}
\end{Shaded}
\end{samepage}

\begin{samepage}
\begin{Shaded}
\begin{Highlighting}[]
\NormalTok{  Yes:}
\NormalTok{  long\_string }\OperatorTok{=}\NormalTok{ (}\StringTok{"And this too is fine if you cannot accept}\CharTok{\textbackslash{}n}\StringTok{"}
                 \StringTok{"extraneous leading spaces."}\NormalTok{)}
\end{Highlighting}
\end{Shaded}
\end{samepage}

\begin{samepage}
\begin{Shaded}
\begin{Highlighting}[]
\NormalTok{  Yes:}
  \ImportTok{import}\NormalTok{ textwrap}

\NormalTok{  long\_string }\OperatorTok{=}\NormalTok{ textwrap.dedent(}\StringTok{"""}\CharTok{\textbackslash{}}
\StringTok{      This is also fine, because textwrap.dedent()}
\StringTok{      will collapse common leading spaces in each line."""}\NormalTok{)}
\end{Highlighting}
\end{Shaded}
\end{samepage}

Note that using a backslash here does not violate the prohibition
against \hyperref[line-length]{explicit line continuation}; in this
case, the backslash is
\href{https://docs.python.org/3/reference/lexical_analysis.html\#string-and-bytes-literals}{escaping
a newline} in a string literal.

\paragraph{Logging}

For logging functions that expect a pattern-string (with
\%-placeholders) as their first argument: Always call them with a string
literal (not an f-string!) as their first argument with
pattern-parameters as subsequent arguments. Some logging implementations
collect the unexpanded pattern-string as a queryable field. It also
prevents spending time rendering a message that no logger is configured
to output.

\begin{samepage}
\begin{Shaded}
\begin{Highlighting}[]
\NormalTok{  Yes:}
  \ImportTok{import}\NormalTok{ tensorflow }\ImportTok{as}\NormalTok{ tf}
\NormalTok{  logger }\OperatorTok{=}\NormalTok{ tf.get\_logger()}
\NormalTok{  logger.info(}\StringTok{\textquotesingle{}TensorFlow Version is: }\SpecialCharTok{\%s}\StringTok{\textquotesingle{}}\NormalTok{, tf.\_\_version\_\_)}
\end{Highlighting}
\end{Shaded}
\end{samepage}

\begin{samepage}
\begin{Shaded}
\begin{Highlighting}[]
\NormalTok{  Yes:}
  \ImportTok{import}\NormalTok{ os}
  \ImportTok{from}\NormalTok{ absl }\ImportTok{import}\NormalTok{ logging}

\NormalTok{  logging.info(}\StringTok{\textquotesingle{}Current $PAGER is: }\SpecialCharTok{\%s}\StringTok{\textquotesingle{}}\NormalTok{, os.getenv(}\StringTok{\textquotesingle{}PAGER\textquotesingle{}}\NormalTok{, default}\OperatorTok{=}\StringTok{\textquotesingle{}\textquotesingle{}}\NormalTok{))}

\NormalTok{  homedir }\OperatorTok{=}\NormalTok{ os.getenv(}\StringTok{\textquotesingle{}HOME\textquotesingle{}}\NormalTok{)}
  \ControlFlowTok{if}\NormalTok{ homedir }\KeywordTok{is} \VariableTok{None} \KeywordTok{or} \KeywordTok{not}\NormalTok{ os.access(homedir, os.W\_OK):}
\NormalTok{    logging.error(}\StringTok{\textquotesingle{}Cannot write to home directory, $HOME=}\SpecialCharTok{\%r}\StringTok{\textquotesingle{}}\NormalTok{, homedir)}
\end{Highlighting}
\end{Shaded}
\end{samepage}

\begin{samepage}
\begin{Shaded}
\begin{Highlighting}[]
\NormalTok{  No:}
  \ImportTok{import}\NormalTok{ os}
  \ImportTok{from}\NormalTok{ absl }\ImportTok{import}\NormalTok{ logging}

\NormalTok{  logging.info(}\StringTok{\textquotesingle{}Current $PAGER is:\textquotesingle{}}\NormalTok{)}
\NormalTok{  logging.info(os.getenv(}\StringTok{\textquotesingle{}PAGER\textquotesingle{}}\NormalTok{, default}\OperatorTok{=}\StringTok{\textquotesingle{}\textquotesingle{}}\NormalTok{))}

\NormalTok{  homedir }\OperatorTok{=}\NormalTok{ os.getenv(}\StringTok{\textquotesingle{}HOME\textquotesingle{}}\NormalTok{)}
  \ControlFlowTok{if}\NormalTok{ homedir }\KeywordTok{is} \VariableTok{None} \KeywordTok{or} \KeywordTok{not}\NormalTok{ os.access(homedir, os.W\_OK):}
\NormalTok{    logging.error(}\SpecialStringTok{f\textquotesingle{}Cannot write to home directory, $HOME=}\SpecialCharTok{\{}\NormalTok{homedir}\SpecialCharTok{!r\}}\SpecialStringTok{\textquotesingle{}}\NormalTok{)}
\end{Highlighting}
\end{Shaded}
\end{samepage}

\paragraph{Error Messages}

Error messages (such as: message strings on exceptions like
\texttt{ValueError}, or messages shown to the user) should follow three
guidelines:

\begin{enumerate}
\def\labelenumi{\arabic{enumi}.}
\item
  The message needs to precisely match the actual error condition.
\item
  Interpolated pieces need to always be clearly identifiable as such.
\item
  They should allow simple automated processing (e.g.~grepping).
\end{enumerate}

\begin{samepage}
\begin{Shaded}
\begin{Highlighting}[]
\NormalTok{  Yes:}
  \ControlFlowTok{if} \KeywordTok{not} \DecValTok{0} \OperatorTok{\textless{}=}\NormalTok{ p }\OperatorTok{\textless{}=} \DecValTok{1}\NormalTok{:}
    \ControlFlowTok{raise} \PreprocessorTok{ValueError}\NormalTok{(}\SpecialStringTok{f\textquotesingle{}Not a probability: }\SpecialCharTok{\{}\NormalTok{p}\OperatorTok{=}\SpecialCharTok{\}}\SpecialStringTok{\textquotesingle{}}\NormalTok{)}

  \ControlFlowTok{try}\NormalTok{:}
\NormalTok{    os.rmdir(workdir)}
  \ControlFlowTok{except} \PreprocessorTok{OSError} \ImportTok{as}\NormalTok{ error:}
\NormalTok{    logging.warning(}\StringTok{\textquotesingle{}Could not remove directory (reason: }\SpecialCharTok{\%r}\StringTok{): }\SpecialCharTok{\%r}\StringTok{\textquotesingle{}}\NormalTok{,}
\NormalTok{                    error, workdir)}
\end{Highlighting}
\end{Shaded}
\end{samepage}

\begin{samepage}
\begin{Shaded}
\begin{Highlighting}[]
\NormalTok{  No:}
  \ControlFlowTok{if}\NormalTok{ p }\OperatorTok{\textless{}} \DecValTok{0} \KeywordTok{or}\NormalTok{ p }\OperatorTok{\textgreater{}} \DecValTok{1}\NormalTok{:  }\CommentTok{\# PROBLEM: also false for float(\textquotesingle{}nan\textquotesingle{})!}
    \ControlFlowTok{raise} \PreprocessorTok{ValueError}\NormalTok{(}\SpecialStringTok{f\textquotesingle{}Not a probability: }\SpecialCharTok{\{}\NormalTok{p}\OperatorTok{=}\SpecialCharTok{\}}\SpecialStringTok{\textquotesingle{}}\NormalTok{)}

  \ControlFlowTok{try}\NormalTok{:}
\NormalTok{    os.rmdir(workdir)}
  \ControlFlowTok{except} \PreprocessorTok{OSError}\NormalTok{:}
    \CommentTok{\# PROBLEM: Message makes an assumption that might not be true:}
    \CommentTok{\# Deletion might have failed for some other reason, misleading}
    \CommentTok{\# whoever has to debug this.}
\NormalTok{    logging.warning(}\StringTok{\textquotesingle{}Directory already was deleted: }\SpecialCharTok{\%s}\StringTok{\textquotesingle{}}\NormalTok{, workdir)}

  \ControlFlowTok{try}\NormalTok{:}
\NormalTok{    os.rmdir(workdir)}
  \ControlFlowTok{except} \PreprocessorTok{OSError}\NormalTok{:}
    \CommentTok{\# PROBLEM: The message is harder to grep for than necessary, and}
    \CommentTok{\# not universally non{-}confusing for all possible values of \textasciigrave{}workdir\textasciigrave{}.}
    \CommentTok{\# Imagine someone calling a library function with such code}
    \CommentTok{\# using a name such as workdir = \textquotesingle{}deleted\textquotesingle{}. The warning would read:}
    \CommentTok{\# "The deleted directory could not be deleted."}
\NormalTok{    logging.warning(}\StringTok{\textquotesingle{}The }\SpecialCharTok{\%s}\StringTok{ directory could not be deleted.\textquotesingle{}}\NormalTok{, workdir)}
\end{Highlighting}
\end{Shaded}
\end{samepage}

\subsection{Files, Sockets, and similar Stateful Resources}

Explicitly close files and sockets when done with them. This rule
naturally extends to closeable resources that internally use sockets,
such as database connections, and also other resources that need to be
closed down in a similar fashion. To name only a few examples, this also
includes \href{https://docs.python.org/3/library/mmap.html}{mmap}
mappings, \href{https://docs.h5py.org/en/stable/high/file.html}{h5py
File objects}, and
\href{https://matplotlib.org/2.1.0/api/_as_gen/matplotlib.pyplot.close.html}{matplotlib.pyplot
figure windows}.

Leaving files, sockets or other such stateful objects open unnecessarily
has many downsides:

\begin{itemize}
\tightlist
\item
  They may consume limited system resources, such as file descriptors.
  Code that deals with many such objects may exhaust those resources
  unnecessarily if they're not returned to the system promptly after
  use.
\item
  Holding files open may prevent other actions such as moving or
  deleting them, or unmounting a filesystem.
\item
  Files and sockets that are shared throughout a program may
  inadvertently be read from or written to after logically being closed.
  If they are actually closed, attempts to read or write from them will
  raise exceptions, making the problem known sooner.
\end{itemize}

Furthermore, while files and sockets (and some similarly behaving
resources) are automatically closed when the object is destructed,
coupling the lifetime of the object to the state of the resource is poor
practice:

\begin{itemize}
\tightlist
\item
  There are no guarantees as to when the runtime will actually invoke
  the \texttt{\_\_del\_\_} method. Different Python implementations use
  different memory management techniques, such as delayed garbage
  collection, which may increase the object's lifetime arbitrarily and
  indefinitely.
\item
  Unexpected references to the file, e.g.~in globals or exception
  tracebacks, may keep it around longer than intended.
\end{itemize}

Relying on finalizers to do automatic cleanup that has observable side
effects has been rediscovered over and over again to lead to major
problems, across many decades and multiple languages (see e.g.
\href{https://wiki.sei.cmu.edu/confluence/display/java/MET12-J.+Do+not+use+finalizers}{this
article} for Java).

The preferred way to manage files and similar resources is using the
\href{http://docs.python.org/reference/compound_stmts.html\#the-with-statement}{\texttt{with}
statement}:

\begin{samepage}
\begin{Shaded}
\begin{Highlighting}[]
\ControlFlowTok{with} \BuiltInTok{open}\NormalTok{(}\StringTok{"hello.txt"}\NormalTok{) }\ImportTok{as}\NormalTok{ hello\_file:}
    \ControlFlowTok{for}\NormalTok{ line }\KeywordTok{in}\NormalTok{ hello\_file:}
        \BuiltInTok{print}\NormalTok{(line)}
\end{Highlighting}
\end{Shaded}
\end{samepage}

For file-like objects that do not support the \texttt{with} statement,
use \texttt{contextlib.closing()}:

\begin{samepage}
\begin{Shaded}
\begin{Highlighting}[]
\ImportTok{import}\NormalTok{ contextlib}

\ControlFlowTok{with}\NormalTok{ contextlib.closing(urllib.urlopen(}\StringTok{"http://www.python.org/"}\NormalTok{)) }\ImportTok{as}\NormalTok{ front\_page:}
    \ControlFlowTok{for}\NormalTok{ line }\KeywordTok{in}\NormalTok{ front\_page:}
        \BuiltInTok{print}\NormalTok{(line)}
\end{Highlighting}
\end{Shaded}
\end{samepage}

In rare cases where context-based resource management is infeasible,
code documentation must explain clearly how resource lifetime is
managed.

\subsection{TODO Comments}

Use \texttt{TODO} comments for code that is temporary, a short-term
solution, or good-enough but not perfect.

A \texttt{TODO} comment begins with the word \texttt{TODO} in all caps,
a following colon, and a link to a resource that contains the context,
ideally a bug reference. A bug reference is preferable because bugs are
tracked and have follow-up comments. Follow this piece of context with
an explanatory string introduced with a hyphen \texttt{-}. The purpose
is to have a consistent \texttt{TODO} format that can be searched to
find out how to get more details.

\begin{samepage}
\begin{Shaded}
\begin{Highlighting}[]
\CommentTok{\# }\AlertTok{TODO}\CommentTok{: crbug.com/192795 {-} Investigate cpufreq optimizations.}
\end{Highlighting}
\end{Shaded}
\end{samepage}

Old style, formerly recommended, but discouraged for use in new code:

\begin{samepage}
\begin{Shaded}
\begin{Highlighting}[]
\CommentTok{\# }\AlertTok{TODO}\CommentTok{(crbug.com/192795): Investigate cpufreq optimizations.}
\CommentTok{\# }\AlertTok{TODO}\CommentTok{(yourusername): Use a "\textbackslash{}*" here for concatenation operator.}
\end{Highlighting}
\end{Shaded}
\end{samepage}

Avoid adding TODOs that refer to an individual or team as the context:

\begin{samepage}
\begin{Shaded}
\begin{Highlighting}[]
\CommentTok{\# }\AlertTok{TODO}\CommentTok{: @yourusername {-} File an issue and use a \textquotesingle{}*\textquotesingle{} for repetition.}
\end{Highlighting}
\end{Shaded}
\end{samepage}

If your \texttt{TODO} is of the form ``At a future date do something''
make sure that you either include a very specific date (``Fix by
November 2009'') or a very specific event (``Remove this code when all
clients can handle XML responses.'') that future code maintainers will
comprehend. Issues are ideal for tracking this.

\subsection{Imports formatting}

Imports should be on separate lines; there are
\hyperref[typing-imports]{exceptions for \texttt{typing} and
\texttt{collections.abc} imports}.

E.g.:

\begin{samepage}
\begin{Shaded}
\begin{Highlighting}[]
\NormalTok{Yes: }\ImportTok{from}\NormalTok{ collections.abc }\ImportTok{import}\NormalTok{ Mapping, Sequence}
     \ImportTok{import}\NormalTok{ os}
     \ImportTok{import}\NormalTok{ sys}
     \ImportTok{from}\NormalTok{ typing }\ImportTok{import}\NormalTok{ Any, NewType}
\end{Highlighting}
\end{Shaded}
\end{samepage}

\begin{samepage}
\begin{Shaded}
\begin{Highlighting}[]
\NormalTok{No:  }\ImportTok{import}\NormalTok{ os, sys}
\end{Highlighting}
\end{Shaded}
\end{samepage}

Imports are always put at the top of the file, just after any module
comments and docstrings and before module globals and constants. Imports
should be grouped from most generic to least generic:

\begin{enumerate}
\def\labelenumi{\arabic{enumi}.}
\item
  Python future import statements. For example:

\begin{samepage}
  \begin{Shaded}
\begin{Highlighting}[]
\ImportTok{from}\NormalTok{ \_\_future\_\_ }\ImportTok{import}\NormalTok{ annotations}
\end{Highlighting}
\end{Shaded}
\end{samepage}

  See \hyperref[from-future-imports]{above} for more information about
  those.
\item
  Python standard library imports. For example:

\begin{samepage}
  \begin{Shaded}
\begin{Highlighting}[]
\ImportTok{import}\NormalTok{ sys}
\end{Highlighting}
\end{Shaded}
\end{samepage}
\item
  \href{https://pypi.org/}{third-party} module or package imports. For
  example:

\begin{samepage}
  \begin{Shaded}
\begin{Highlighting}[]
\ImportTok{import}\NormalTok{ tensorflow }\ImportTok{as}\NormalTok{ tf}
\end{Highlighting}
\end{Shaded}
\end{samepage}
\item
  Code repository sub-package imports. For example:

\begin{samepage}
  \begin{Shaded}
\begin{Highlighting}[]
\ImportTok{from}\NormalTok{ otherproject.ai }\ImportTok{import}\NormalTok{ mind}
\end{Highlighting}
\end{Shaded}
\end{samepage}
\item
  \textbf{Deprecated:} application-specific imports that are part of the
  same top-level sub-package as this file. For example:

\begin{samepage}
  \begin{Shaded}
\begin{Highlighting}[]
\ImportTok{from}\NormalTok{ myproject.backend.hgwells }\ImportTok{import}\NormalTok{ time\_machine}
\end{Highlighting}
\end{Shaded}
\end{samepage}

  You may find older Google Python Style code doing this, but it is no
  longer required. \textbf{New code is encouraged not to bother with
  this.} Simply treat application-specific sub-package imports the same
  as other sub-package imports.
\end{enumerate}

Within each grouping, imports should be sorted lexicographically,
ignoring case, according to each module's full package path (the
\texttt{path} in \texttt{from\ path\ import\ ...}). Code may optionally
place a blank line between import sections.

\begin{samepage}
\begin{Shaded}
\begin{Highlighting}[]
\ImportTok{import}\NormalTok{ collections}
\ImportTok{import}\NormalTok{ queue}
\ImportTok{import}\NormalTok{ sys}

\ImportTok{from}\NormalTok{ absl }\ImportTok{import}\NormalTok{ app}
\ImportTok{from}\NormalTok{ absl }\ImportTok{import}\NormalTok{ flags}
\ImportTok{import}\NormalTok{ bs4}
\ImportTok{import}\NormalTok{ cryptography}
\ImportTok{import}\NormalTok{ tensorflow }\ImportTok{as}\NormalTok{ tf}

\ImportTok{from}\NormalTok{ book.genres }\ImportTok{import}\NormalTok{ scifi}
\ImportTok{from}\NormalTok{ myproject.backend }\ImportTok{import}\NormalTok{ huxley}
\ImportTok{from}\NormalTok{ myproject.backend.hgwells }\ImportTok{import}\NormalTok{ time\_machine}
\ImportTok{from}\NormalTok{ myproject.backend.state\_machine }\ImportTok{import}\NormalTok{ main\_loop}
\ImportTok{from}\NormalTok{ otherproject.ai }\ImportTok{import}\NormalTok{ body}
\ImportTok{from}\NormalTok{ otherproject.ai }\ImportTok{import}\NormalTok{ mind}
\ImportTok{from}\NormalTok{ otherproject.ai }\ImportTok{import}\NormalTok{ soul}

\CommentTok{\# Older style code may have these imports down here instead:}
\CommentTok{\#from myproject.backend.hgwells import time\_machine}
\CommentTok{\#from myproject.backend.state\_machine import main\_loop}
\end{Highlighting}
\end{Shaded}
\end{samepage}

\subsection{Statements}

Generally only one statement per line.

However, you may put the result of a test on the same line as the test
only if the entire statement fits on one line. In particular, you can
never do so with \texttt{try}/\texttt{except} since the \texttt{try} and
\texttt{except} can't both fit on the same line, and you can only do so
with an \texttt{if} if there is no \texttt{else}.

\begin{samepage}
\begin{Shaded}
\begin{Highlighting}[]
\NormalTok{Yes:}

  \ControlFlowTok{if}\NormalTok{ foo: bar(foo)}
\end{Highlighting}
\end{Shaded}
\end{samepage}

\begin{samepage}
\begin{Shaded}
\begin{Highlighting}[]
\NormalTok{No:}

  \ControlFlowTok{if}\NormalTok{ foo: bar(foo)}
  \ControlFlowTok{else}\NormalTok{:   baz(foo)}

  \ControlFlowTok{try}\NormalTok{:               bar(foo)}
  \ControlFlowTok{except} \PreprocessorTok{ValueError}\NormalTok{: baz(foo)}

  \ControlFlowTok{try}\NormalTok{:}
\NormalTok{      bar(foo)}
  \ControlFlowTok{except} \PreprocessorTok{ValueError}\NormalTok{: baz(foo)}
\end{Highlighting}
\end{Shaded}
\end{samepage}

\subsection{Getters and Setters}

Getter and setter functions (also called accessors and mutators) should
be used when they provide a meaningful role or behavior for getting or
setting a variable's value.

In particular, they should be used when getting or setting the variable
is complex or the cost is significant, either currently or in a
reasonable future.

If, for example, a pair of getters/setters simply read and write an
internal attribute, the internal attribute should be made public
instead. By comparison, if setting a variable means some state is
invalidated or rebuilt, it should be a setter function. The function
invocation hints that a potentially non-trivial operation is occurring.
Alternatively, \hyperref[properties]{properties} may be an option when
simple logic is needed, or refactoring to no longer need getters and
setters.

Getters and setters should follow the \hyperref[s3.16-naming]{Naming}
guidelines, such as \texttt{get\_foo()} and \texttt{set\_foo()}.

If the past behavior allowed access through a property, do not bind the
new getter/setter functions to the property. Any code still attempting
to access the variable by the old method should break visibly so they
are made aware of the change in complexity.

\subsection{Naming}

\texttt{module\_name}, \texttt{package\_name}, \texttt{ClassName},
\texttt{method\_name}, \texttt{ExceptionName}, \texttt{function\_name},
\texttt{GLOBAL\_CONSTANT\_NAME}, \texttt{global\_var\_name},
\texttt{instance\_var\_name}, \texttt{function\_parameter\_name},
\texttt{local\_var\_name}, \texttt{query\_proper\_noun\_for\_thing},
\texttt{send\_acronym\_via\_https}.

Function names, variable names, and filenames should be descriptive;
avoid abbreviation. In particular, do not use abbreviations that are
ambiguous or unfamiliar to readers outside your project, and do not
abbreviate by deleting letters within a word.

Always use a \texttt{.py} filename extension. Never use dashes.

\paragraph{Names to Avoid}

\begin{itemize}
\item
  single character names, except for specifically allowed cases:

  \begin{itemize}
  \tightlist
  \item
    counters or iterators (e.g.~\texttt{i}, \texttt{j}, \texttt{k},
    \texttt{v}, et al.)
  \item
    \texttt{e} as an exception identifier in \texttt{try/except}
    statements.
  \item
    \texttt{f} as a file handle in \texttt{with} statements
  \item
    private \hyperref[typing-type-var]{type variables} with no
    constraints (e.g. \texttt{\_T\ =\ TypeVar("\_T")},
    \texttt{\_P\ =\ ParamSpec("\_P")})
  \end{itemize}

  Please be mindful not to abuse single-character naming. Generally
  speaking, descriptiveness should be proportional to the name's scope
  of visibility. For example, \texttt{i} might be a fine name for 5-line
  code block but within multiple nested scopes, it is likely too vague.
\item
  dashes (\texttt{-}) in any package/module name
\item
  \texttt{\_\_double\_leading\_and\_trailing\_underscore\_\_} names
  (reserved by Python)
\item
  offensive terms
\item
  names that needlessly include the type of the variable (for example:
  \texttt{id\_to\_name\_dict})
\end{itemize}

\paragraph{Naming Conventions}

\begin{itemize}
\item
  ``Internal'' means internal to a module, or protected or private
  within a class.
\item
  Prepending a single underscore (\texttt{\_}) has some support for
  protecting module variables and functions (linters will flag protected
  member access). Note that it is okay for unit tests to access
  protected constants from the modules under test.
\item
  Prepending a double underscore (\texttt{\_\_} aka ``dunder'') to an
  instance variable or method effectively makes the variable or method
  private to its class (using name mangling); we discourage its use as
  it impacts readability and testability, and isn't \emph{really}
  private. Prefer a single underscore.
\item
  Place related classes and top-level functions together in a module.
  Unlike Java, there is no need to limit yourself to one class per
  module.
\item
  Use CapWords for class names, but lower\_with\_under.py for module
  names. Although there are some old modules named CapWords.py, this is
  now discouraged because it's confusing when the module happens to be
  named after a class. (``wait -- did I write \texttt{import\ StringIO}
  or \texttt{from\ StringIO\ import\ \ \ \ \ StringIO}?'')
\item
  New \emph{unit test} files follow PEP 8 compliant lower\_with\_under
  method names, for example,
  \texttt{test\_\textless{}method\_under\_test\textgreater{}\_\textless{}state\textgreater{}}.
  For consistency(*) with legacy modules that follow CapWords function
  names, underscores may appear in method names starting with
  \texttt{test} to separate logical components of the name. One possible
  pattern is
  \texttt{test\textless{}MethodUnderTest\textgreater{}\_\textless{}state\textgreater{}}.
\end{itemize}

\paragraph{File Naming}

Python filenames must have a \texttt{.py} extension and must not contain
dashes (\texttt{-}). This allows them to be imported and unittested. If
you want an executable to be accessible without the extension, use a
symbolic link or a simple bash wrapper containing
\texttt{exec\ "\$0.py"\ "\$@"}.

\paragraph{Guidelines derived from}
\href{https://en.wikipedia.org/wiki/Guido_van_Rossum}{Guido}'s
\begin{table}[h!]
  \centering
  \begin{tabular}{|>{\raggedright\arraybackslash}p{4cm}|>{\raggedright\arraybackslash}p{4cm}|>{\raggedright\arraybackslash}p{4cm}|}
  \hline
  \textbf{Type} & \textbf{Public} & \textbf{Internal} \\
  \hline
  Packages & \texttt{lower\_with\_under} &  \\
  \hline
  Modules & \texttt{lower\_with\_under} & \texttt{\_lower\_with\_under} \\
  \hline
  Classes & \texttt{CapWords} & \texttt{\_CapWords} \\
  \hline
  Exceptions & \texttt{CapWords} &  \\
  \hline
  Functions & \texttt{lower\_with\_under()} & \texttt{\_lower\_with\_under()} \\
  \hline
  Global/Class Constants & \texttt{CAPS\_WITH\_UNDER} & \texttt{\_CAPS\_WITH\_UNDER} \\
  \hline
  Global/Class Variables & \texttt{lower\_with\_under} & \texttt{\_lower\_with\_under} \\
  \hline
  Instance Variables & \texttt{lower\_with\_under} & \texttt{\_lower\_with\_under} (protected) \\
  \hline
  Method Names & \texttt{lower\_with\_under()} & \texttt{\_lower\_with\_under()} (protected) \\
  \hline
  Function/Method Parameters & \texttt{lower\_with\_under} &  \\
  \hline
  Local Variables & \texttt{lower\_with\_under} &  \\
  \hline
  \end{tabular}
  \caption{Guidelines from Guido's Recommendations}
  \label{tab:guidelines}
  \end{table}

\paragraph{Mathematical Notation}

For mathematically heavy code, short variable names that would otherwise
violate the style guide are preferred when they match established
notation in a reference paper or algorithm. When doing so, reference the
source of all naming conventions in a comment or docstring or, if the
source is not accessible, clearly document the naming conventions.
Prefer PEP8-compliant \texttt{descriptive\_names} for public APIs, which
are much more likely to be encountered out of context.

\subsection{Main}

In Python, \texttt{pydoc} as well as unit tests require modules to be
importable. If a file is meant to be used as an executable, its main
functionality should be in a \texttt{main()} function, and your code
should always check
\texttt{if\ \_\_name\_\_\ ==\ \textquotesingle{}\_\_main\_\_\textquotesingle{}}
before executing your main program, so that it is not executed when the
module is imported.

When using \href{https://github.com/abseil/abseil-py}{absl}, use
\texttt{app.run}:

\begin{samepage}
\begin{Shaded}
\begin{Highlighting}[]
\ImportTok{from}\NormalTok{ absl }\ImportTok{import}\NormalTok{ app}
\NormalTok{...}

\KeywordTok{def}\NormalTok{ main(argv: Sequence[}\BuiltInTok{str}\NormalTok{]):}
    \CommentTok{\# process non{-}flag arguments}
\NormalTok{    ...}

\ControlFlowTok{if} \VariableTok{\_\_name\_\_} \OperatorTok{==} \StringTok{\textquotesingle{}\_\_main\_\_\textquotesingle{}}\NormalTok{:}
\NormalTok{    app.run(main)}
\end{Highlighting}
\end{Shaded}
\end{samepage}

Otherwise, use:

\begin{samepage}
\begin{Shaded}
\begin{Highlighting}[]
\KeywordTok{def}\NormalTok{ main():}
\NormalTok{    ...}

\ControlFlowTok{if} \VariableTok{\_\_name\_\_} \OperatorTok{==} \StringTok{\textquotesingle{}\_\_main\_\_\textquotesingle{}}\NormalTok{:}
\NormalTok{    main()}
\end{Highlighting}
\end{Shaded}
\end{samepage}

All code at the top level will be executed when the module is imported.
Be careful not to call functions, create objects, or perform other
operations that should not be executed when the file is being
\texttt{pydoc}ed.

\subsection{Function length}

Prefer small and focused functions.

We recognize that long functions are sometimes appropriate, so no hard
limit is placed on function length. If a function exceeds about 40
lines, think about whether it can be broken up without harming the
structure of the program.

Even if your long function works perfectly now, someone modifying it in
a few months may add new behavior. This could result in bugs that are
hard to find. Keeping your functions short and simple makes it easier
for other people to read and modify your code.

You could find long and complicated functions when working with some
code. Do not be intimidated by modifying existing code: if working with
such a function proves to be difficult, you find that errors are hard to
debug, or you want to use a piece of it in several different contexts,
consider breaking up the function into smaller and more manageable
pieces.

\subsection{Type Annotations}

\paragraph{General Rules}

\begin{itemize}
\item
  Familiarize yourself with
  \href{https://peps.python.org/pep-0484/}{PEP-484}.
\item
  Annotating \texttt{self} or \texttt{cls} is generally not necessary.
  \href{https://docs.python.org/3/library/typing.html\#typing.Self}{\texttt{Self}}
  can be used if it is necessary for proper type information, e.g.

\begin{samepage}
  \begin{Shaded}
\begin{Highlighting}[]
\ImportTok{from}\NormalTok{ typing }\ImportTok{import}\NormalTok{ Self}

\KeywordTok{class}\NormalTok{ BaseClass:}
  \AttributeTok{@classmethod}
  \KeywordTok{def}\NormalTok{ create(cls) }\OperatorTok{{-}\textgreater{}}\NormalTok{ Self:}
\NormalTok{    ...}

  \KeywordTok{def}\NormalTok{ difference(}\VariableTok{self}\NormalTok{, other: Self) }\OperatorTok{{-}\textgreater{}} \BuiltInTok{float}\NormalTok{:}
\NormalTok{    ...}
\end{Highlighting}
\end{Shaded}
\end{samepage}
\item
  Similarly, don't feel compelled to annotate the return value of
  \texttt{\_\_init\_\_} (where \texttt{None} is the only valid option).
\item
  If any other variable or a returned type should not be expressed, use
  \texttt{Any}.
\item
  You are not required to annotate all the functions in a module.

  \begin{itemize}
  \tightlist
  \item
    At least annotate your public APIs.
  \item
    Use judgment to get to a good balance between safety and clarity on
    the one hand, and flexibility on the other.
  \item
    Annotate code that is prone to type-related errors (previous bugs or
    complexity).
  \item
    Annotate code that is hard to understand.
  \item
    Annotate code as it becomes stable from a types perspective. In many
    cases, you can annotate all the functions in mature code without
    losing too much flexibility.
  \end{itemize}
\end{itemize}

\paragraph{Line Breaking}

Try to follow the existing \hyperref[indentation]{indentation} rules.

After annotating, many function signatures will become ``one parameter
per line''. To ensure the return type is also given its own line, a
comma can be placed after the last parameter.

\begin{samepage}
\begin{Shaded}
\begin{Highlighting}[]
\KeywordTok{def}\NormalTok{ my\_method(}
    \VariableTok{self}\NormalTok{,}
\NormalTok{    first\_var: }\BuiltInTok{int}\NormalTok{,}
\NormalTok{    second\_var: Foo,}
\NormalTok{    third\_var: Bar }\OperatorTok{|} \VariableTok{None}\NormalTok{,}
\NormalTok{) }\OperatorTok{{-}\textgreater{}} \BuiltInTok{int}\NormalTok{:}
\NormalTok{  ...}
\end{Highlighting}
\end{Shaded}
\end{samepage}

Always prefer breaking between variables, and not, for example, between
variable names and type annotations. However, if everything fits on the
same line, go for it.

\begin{samepage}
\begin{Shaded}
\begin{Highlighting}[]
\KeywordTok{def}\NormalTok{ my\_method(}\VariableTok{self}\NormalTok{, first\_var: }\BuiltInTok{int}\NormalTok{) }\OperatorTok{{-}\textgreater{}} \BuiltInTok{int}\NormalTok{:}
\NormalTok{  ...}
\end{Highlighting}
\end{Shaded}
\end{samepage}

If the combination of the function name, the last parameter, and the
return type is too long, indent by 4 in a new line. When using line
breaks, prefer putting each parameter and the return type on their own
lines and aligning the closing parenthesis with the \texttt{def}:

\begin{samepage}
\begin{Shaded}
\begin{Highlighting}[]
\NormalTok{Yes:}
\KeywordTok{def}\NormalTok{ my\_method(}
    \VariableTok{self}\NormalTok{,}
\NormalTok{    other\_arg: MyLongType }\OperatorTok{|} \VariableTok{None}\NormalTok{,}
\NormalTok{) }\OperatorTok{{-}\textgreater{}} \BuiltInTok{tuple}\NormalTok{[MyLongType1, MyLongType1]:}
\NormalTok{  ...}
\end{Highlighting}
\end{Shaded}
\end{samepage}

Optionally, the return type may be put on the same line as the last
parameter:

\begin{samepage}
\begin{Shaded}
\begin{Highlighting}[]
\NormalTok{Okay:}
\KeywordTok{def}\NormalTok{ my\_method(}
    \VariableTok{self}\NormalTok{,}
\NormalTok{    first\_var: }\BuiltInTok{int}\NormalTok{,}
\NormalTok{    second\_var: }\BuiltInTok{int}\NormalTok{) }\OperatorTok{{-}\textgreater{}} \BuiltInTok{dict}\NormalTok{[OtherLongType, MyLongType]:}
\NormalTok{  ...}
\end{Highlighting}
\end{Shaded}
\end{samepage}

\texttt{pylint} allows you to move the closing parenthesis to a new line
and align with the opening one, but this is less readable.

\begin{samepage}
\begin{Shaded}
\begin{Highlighting}[]
\NormalTok{No:}
\KeywordTok{def}\NormalTok{ my\_method(}\VariableTok{self}\NormalTok{,}
\NormalTok{              other\_arg: MyLongType }\OperatorTok{|} \VariableTok{None}\NormalTok{,}
\NormalTok{             ) }\OperatorTok{{-}\textgreater{}} \BuiltInTok{dict}\NormalTok{[OtherLongType, MyLongType]:}
\NormalTok{  ...}
\end{Highlighting}
\end{Shaded}
\end{samepage}

As in the examples above, prefer not to break types. However, sometimes
they are too long to be on a single line (try to keep sub-types
unbroken).

\begin{samepage}
\begin{Shaded}
\begin{Highlighting}[]
\KeywordTok{def}\NormalTok{ my\_method(}
    \VariableTok{self}\NormalTok{,}
\NormalTok{    first\_var: }\BuiltInTok{tuple}\NormalTok{[}\BuiltInTok{list}\NormalTok{[MyLongType1],}
                     \BuiltInTok{list}\NormalTok{[MyLongType2]],}
\NormalTok{    second\_var: }\BuiltInTok{list}\NormalTok{[}\BuiltInTok{dict}\NormalTok{[}
\NormalTok{        MyLongType3, MyLongType4]],}
\NormalTok{) }\OperatorTok{{-}\textgreater{}} \VariableTok{None}\NormalTok{:}
\NormalTok{  ...}
\end{Highlighting}
\end{Shaded}
\end{samepage}

If a single name and type is too long, consider using an
\hyperref[typing-aliases]{alias} for the type. The last resort is to
break after the colon and indent by 4.

\begin{samepage}
\begin{Shaded}
\begin{Highlighting}[]
\NormalTok{Yes:}
\KeywordTok{def}\NormalTok{ my\_function(}
\NormalTok{    long\_variable\_name:}
\NormalTok{        long\_module\_name.LongTypeName,}
\NormalTok{) }\OperatorTok{{-}\textgreater{}} \VariableTok{None}\NormalTok{:}
\NormalTok{  ...}
\end{Highlighting}
\end{Shaded}
\end{samepage}

\begin{samepage}
\begin{Shaded}
\begin{Highlighting}[]
\NormalTok{No:}
\KeywordTok{def}\NormalTok{ my\_function(}
\NormalTok{    long\_variable\_name: long\_module\_name.}
\NormalTok{        LongTypeName,}
\NormalTok{) }\OperatorTok{{-}\textgreater{}} \VariableTok{None}\NormalTok{:}
\NormalTok{  ...}
\end{Highlighting}
\end{Shaded}
\end{samepage}

\paragraph{Forward Declarations}

If you need to use a class name (from the same module) that is not yet
defined -- for example, if you need the class name inside the
declaration of that class, or if you use a class that is defined later
in the code -- either use
\texttt{from\ \_\_future\_\_\ import\ annotations} or use a string for
the class name.

\begin{samepage}
\begin{Shaded}
\begin{Highlighting}[]
\NormalTok{Yes:}
\ImportTok{from}\NormalTok{ \_\_future\_\_ }\ImportTok{import}\NormalTok{ annotations}

\KeywordTok{class}\NormalTok{ MyClass:}
  \KeywordTok{def} \FunctionTok{\_\_init\_\_}\NormalTok{(}\VariableTok{self}\NormalTok{, stack: Sequence[MyClass], item: OtherClass) }\OperatorTok{{-}\textgreater{}} \VariableTok{None}\NormalTok{:}

\KeywordTok{class}\NormalTok{ OtherClass:}
\NormalTok{  ...}
\end{Highlighting}
\end{Shaded}
\end{samepage}

\begin{samepage}
\begin{Shaded}
\begin{Highlighting}[]
\NormalTok{Yes:}
\KeywordTok{class}\NormalTok{ MyClass:}
  \KeywordTok{def} \FunctionTok{\_\_init\_\_}\NormalTok{(}\VariableTok{self}\NormalTok{, stack: Sequence[}\StringTok{\textquotesingle{}MyClass\textquotesingle{}}\NormalTok{], item: }\StringTok{\textquotesingle{}OtherClass\textquotesingle{}}\NormalTok{) }\OperatorTok{{-}\textgreater{}} \VariableTok{None}\NormalTok{:}

\KeywordTok{class}\NormalTok{ OtherClass:}
\NormalTok{  ...}
\end{Highlighting}
\end{Shaded}
\end{samepage}

\paragraph{Default Values}

As per
\href{https://peps.python.org/pep-0008/\#other-recommendations}{PEP-008},
use spaces around the \texttt{=} \emph{only} for arguments that have
both a type annotation and a default value.

\begin{samepage}
\begin{Shaded}
\begin{Highlighting}[]
\NormalTok{Yes:}
\KeywordTok{def}\NormalTok{ func(a: }\BuiltInTok{int} \OperatorTok{=} \DecValTok{0}\NormalTok{) }\OperatorTok{{-}\textgreater{}} \BuiltInTok{int}\NormalTok{:}
\NormalTok{  ...}
\end{Highlighting}
\end{Shaded}
\end{samepage}

\begin{samepage}
\begin{Shaded}
\begin{Highlighting}[]
\NormalTok{No:}
\KeywordTok{def}\NormalTok{ func(a:}\BuiltInTok{int}\OperatorTok{=}\DecValTok{0}\NormalTok{) }\OperatorTok{{-}\textgreater{}} \BuiltInTok{int}\NormalTok{:}
\NormalTok{  ...}
\end{Highlighting}
\end{Shaded}
\end{samepage}

\paragraph{NoneType}

In the Python type system, \texttt{NoneType} is a ``first class'' type,
and for typing purposes, \texttt{None} is an alias for
\texttt{NoneType}. If an argument can be \texttt{None}, it has to be
declared! You can use \texttt{\textbar{}} union type expressions
(recommended in new Python 3.10+ code), or the older \texttt{Optional}
and \texttt{Union} syntaxes.

Use explicit \texttt{X\ \textbar{}\ None} instead of implicit. Earlier
versions of PEP 484 allowed \texttt{a:\ str\ =\ None} to be interpreted
as \texttt{a:\ str\ \textbar{}\ None\ =\ None}, but that is no longer
the preferred behavior.

\begin{samepage}
\begin{Shaded}
\begin{Highlighting}[]
\NormalTok{Yes:}
\KeywordTok{def}\NormalTok{ modern\_or\_union(a: }\BuiltInTok{str} \OperatorTok{|} \BuiltInTok{int} \OperatorTok{|} \VariableTok{None}\NormalTok{, b: }\BuiltInTok{str} \OperatorTok{|} \VariableTok{None} \OperatorTok{=} \VariableTok{None}\NormalTok{) }\OperatorTok{{-}\textgreater{}} \BuiltInTok{str}\NormalTok{:}
\NormalTok{  ...}
\KeywordTok{def}\NormalTok{ union\_optional(a: Union[}\BuiltInTok{str}\NormalTok{, }\BuiltInTok{int}\NormalTok{, }\VariableTok{None}\NormalTok{], b: Optional[}\BuiltInTok{str}\NormalTok{] }\OperatorTok{=} \VariableTok{None}\NormalTok{) }\OperatorTok{{-}\textgreater{}} \BuiltInTok{str}\NormalTok{:}
\NormalTok{  ...}
\end{Highlighting}
\end{Shaded}
\end{samepage}

\begin{samepage}
\begin{Shaded}
\begin{Highlighting}[]
\NormalTok{No:}
\KeywordTok{def}\NormalTok{ nullable\_union(a: Union[}\VariableTok{None}\NormalTok{, }\BuiltInTok{str}\NormalTok{]) }\OperatorTok{{-}\textgreater{}} \BuiltInTok{str}\NormalTok{:}
\NormalTok{  ...}
\KeywordTok{def}\NormalTok{ implicit\_optional(a: }\BuiltInTok{str} \OperatorTok{=} \VariableTok{None}\NormalTok{) }\OperatorTok{{-}\textgreater{}} \BuiltInTok{str}\NormalTok{:}
\NormalTok{  ...}
\end{Highlighting}
\end{Shaded}
\end{samepage}

\paragraph{Type Aliases}

You can declare aliases of complex types. The name of an alias should be
CapWorded. If the alias is used only in this module, it should be
\_Private.

Note that the \texttt{:\ TypeAlias} annotation is only supported in
versions 3.10+.

\begin{samepage}
\begin{Shaded}
\begin{Highlighting}[]
\ImportTok{from}\NormalTok{ typing }\ImportTok{import}\NormalTok{ TypeAlias}

\NormalTok{\_LossAndGradient: TypeAlias }\OperatorTok{=} \BuiltInTok{tuple}\NormalTok{[tf.Tensor, tf.Tensor]}
\NormalTok{ComplexTFMap: TypeAlias }\OperatorTok{=}\NormalTok{ Mapping[}\BuiltInTok{str}\NormalTok{, \_LossAndGradient]}
\end{Highlighting}
\end{Shaded}
\end{samepage}

\paragraph{Ignoring Types}

You can disable type checking on a line with the special comment
\texttt{\#\ type:\ ignore}.

\texttt{pytype} has a disable option for specific errors (similar to
lint):

\begin{samepage}
\begin{Shaded}
\begin{Highlighting}[]
\CommentTok{\# pytype: disable=attribute{-}error}
\end{Highlighting}
\end{Shaded}
\end{samepage}

\paragraph{Typing Variables}

\hyperref[annotated-assignments]{\emph{Annotated Assignments}} : If an
internal variable has a type that is hard or impossible to infer,
specify its type with an annotated assignment - use a colon and type
between the variable name and value (the same as is done with function
arguments that have a default value):

\begin{verbatim}
```python
a: Foo = SomeUndecoratedFunction()
```
\end{verbatim}

\hyperref[type-comments]{\emph{Type Comments}} : Though you may see them
remaining in the codebase (they were necessary before Python 3.6), do
not add any more uses of a
\texttt{\#\ type:\ \textless{}type\ name\textgreater{}} comment on the
end of the line:

\begin{verbatim}
```python
a = SomeUndecoratedFunction()  # type: Foo
```
\end{verbatim}

\paragraph{Tuples vs Lists}

Typed lists can only contain objects of a single type. Typed tuples can
either have a single repeated type or a set number of elements with
different types. The latter is commonly used as the return type from a
function.

\begin{samepage}
\begin{Shaded}
\begin{Highlighting}[]
\NormalTok{a: }\BuiltInTok{list}\NormalTok{[}\BuiltInTok{int}\NormalTok{] }\OperatorTok{=}\NormalTok{ [}\DecValTok{1}\NormalTok{, }\DecValTok{2}\NormalTok{, }\DecValTok{3}\NormalTok{]}
\NormalTok{b: }\BuiltInTok{tuple}\NormalTok{[}\BuiltInTok{int}\NormalTok{, ...] }\OperatorTok{=}\NormalTok{ (}\DecValTok{1}\NormalTok{, }\DecValTok{2}\NormalTok{, }\DecValTok{3}\NormalTok{)}
\NormalTok{c: }\BuiltInTok{tuple}\NormalTok{[}\BuiltInTok{int}\NormalTok{, }\BuiltInTok{str}\NormalTok{, }\BuiltInTok{float}\NormalTok{] }\OperatorTok{=}\NormalTok{ (}\DecValTok{1}\NormalTok{, }\StringTok{"2"}\NormalTok{, }\FloatTok{3.5}\NormalTok{)}
\end{Highlighting}
\end{Shaded}
\end{samepage}

\paragraph{3Type variables}

The Python type system has
\href{https://peps.python.org/pep-0484/\#generics}{generics}. A type
variable, such as \texttt{TypeVar} and \texttt{ParamSpec}, is a common
way to use them.

Example:

\begin{samepage}
\begin{Shaded}
\begin{Highlighting}[]
\ImportTok{from}\NormalTok{ collections.abc }\ImportTok{import}\NormalTok{ Callable}
\ImportTok{from}\NormalTok{ typing }\ImportTok{import}\NormalTok{ ParamSpec, TypeVar}
\NormalTok{\_P }\OperatorTok{=}\NormalTok{ ParamSpec(}\StringTok{"\_P"}\NormalTok{)}
\NormalTok{\_T }\OperatorTok{=}\NormalTok{ TypeVar(}\StringTok{"\_T"}\NormalTok{)}
\NormalTok{...}
\KeywordTok{def} \BuiltInTok{next}\NormalTok{(l: }\BuiltInTok{list}\NormalTok{[\_T]) }\OperatorTok{{-}\textgreater{}}\NormalTok{ \_T:}
  \ControlFlowTok{return}\NormalTok{ l.pop()}

\KeywordTok{def}\NormalTok{ print\_when\_called(f: Callable[\_P, \_T]) }\OperatorTok{{-}\textgreater{}}\NormalTok{ Callable[\_P, \_T]:}
  \KeywordTok{def}\NormalTok{ inner(}\OperatorTok{*}\NormalTok{args: \_P.args, }\OperatorTok{**}\NormalTok{kwargs: \_P.kwargs) }\OperatorTok{{-}\textgreater{}}\NormalTok{ \_T:}
    \BuiltInTok{print}\NormalTok{(}\StringTok{"Function was called"}\NormalTok{)}
    \ControlFlowTok{return}\NormalTok{ f(}\OperatorTok{*}\NormalTok{args, }\OperatorTok{**}\NormalTok{kwargs)}
  \ControlFlowTok{return}\NormalTok{ inner}
\end{Highlighting}
\end{Shaded}
\end{samepage}

A \texttt{TypeVar} can be constrained:

\begin{samepage}
\begin{Shaded}
\begin{Highlighting}[]
\NormalTok{AddableType }\OperatorTok{=}\NormalTok{ TypeVar(}\StringTok{"AddableType"}\NormalTok{, }\BuiltInTok{int}\NormalTok{, }\BuiltInTok{float}\NormalTok{, }\BuiltInTok{str}\NormalTok{)}
\KeywordTok{def}\NormalTok{ add(a: AddableType, b: AddableType) }\OperatorTok{{-}\textgreater{}}\NormalTok{ AddableType:}
  \ControlFlowTok{return}\NormalTok{ a }\OperatorTok{+}\NormalTok{ b}
\end{Highlighting}
\end{Shaded}
\end{samepage}

A common predefined type variable in the \texttt{typing} module is
\texttt{AnyStr}. Use it for multiple annotations that can be
\texttt{bytes} or \texttt{str} and must all be the same type.

\begin{samepage}
\begin{Shaded}
\begin{Highlighting}[]
\ImportTok{from}\NormalTok{ typing }\ImportTok{import}\NormalTok{ AnyStr}
\KeywordTok{def}\NormalTok{ check\_length(x: AnyStr) }\OperatorTok{{-}\textgreater{}}\NormalTok{ AnyStr:}
  \ControlFlowTok{if} \BuiltInTok{len}\NormalTok{(x) }\OperatorTok{\textless{}=} \DecValTok{42}\NormalTok{:}
    \ControlFlowTok{return}\NormalTok{ x}
  \ControlFlowTok{raise} \PreprocessorTok{ValueError}\NormalTok{()}
\end{Highlighting}
\end{Shaded}
\end{samepage}

A type variable must have a descriptive name, unless it meets all of the
following criteria:

\begin{itemize}
\tightlist
\item
  not externally visible
\item
  not constrained
\end{itemize}

\begin{samepage}
\begin{Shaded}
\begin{Highlighting}[]
\NormalTok{Yes:}
\NormalTok{  \_T }\OperatorTok{=}\NormalTok{ TypeVar(}\StringTok{"\_T"}\NormalTok{)}
\NormalTok{  \_P }\OperatorTok{=}\NormalTok{ ParamSpec(}\StringTok{"\_P"}\NormalTok{)}
\NormalTok{  AddableType }\OperatorTok{=}\NormalTok{ TypeVar(}\StringTok{"AddableType"}\NormalTok{, }\BuiltInTok{int}\NormalTok{, }\BuiltInTok{float}\NormalTok{, }\BuiltInTok{str}\NormalTok{)}
\NormalTok{  AnyFunction }\OperatorTok{=}\NormalTok{ TypeVar(}\StringTok{"AnyFunction"}\NormalTok{, bound}\OperatorTok{=}\NormalTok{Callable)}
\end{Highlighting}
\end{Shaded}
\end{samepage}

\begin{samepage}
\begin{Shaded}
\begin{Highlighting}[]
\NormalTok{No:}
\NormalTok{  T }\OperatorTok{=}\NormalTok{ TypeVar(}\StringTok{"T"}\NormalTok{)}
\NormalTok{  P }\OperatorTok{=}\NormalTok{ ParamSpec(}\StringTok{"P"}\NormalTok{)}
\NormalTok{  \_T }\OperatorTok{=}\NormalTok{ TypeVar(}\StringTok{"\_T"}\NormalTok{, }\BuiltInTok{int}\NormalTok{, }\BuiltInTok{float}\NormalTok{, }\BuiltInTok{str}\NormalTok{)}
\NormalTok{  \_F }\OperatorTok{=}\NormalTok{ TypeVar(}\StringTok{"\_F"}\NormalTok{, bound}\OperatorTok{=}\NormalTok{Callable)}
\end{Highlighting}
\end{Shaded}
\end{samepage}

\paragraph{String types}

\begin{quote}
Do not use \texttt{typing.Text} in new code. It's only for Python 2/3
compatibility.
\end{quote}

Use \texttt{str} for string/text data. For code that deals with binary
data, use \texttt{bytes}.

\begin{samepage}
\begin{Shaded}
\begin{Highlighting}[]
\KeywordTok{def}\NormalTok{ deals\_with\_text\_data(x: }\BuiltInTok{str}\NormalTok{) }\OperatorTok{{-}\textgreater{}} \BuiltInTok{str}\NormalTok{:}
\NormalTok{  ...}
\KeywordTok{def}\NormalTok{ deals\_with\_binary\_data(x: }\BuiltInTok{bytes}\NormalTok{) }\OperatorTok{{-}\textgreater{}} \BuiltInTok{bytes}\NormalTok{:}
\NormalTok{  ...}
\end{Highlighting}
\end{Shaded}
\end{samepage}

If all the string types of a function are always the same, for example
if the return type is the same as the argument type in the code above,
use \hyperref[typing-type-var]{AnyStr}.

\paragraph{Imports For Typing}

For symbols (including types, functions, and constants) from the
\texttt{typing} or \texttt{collections.abc} modules used to support
static analysis and type checking, always import the symbol itself. This
keeps common annotations more concise and matches typing practices used
around the world. You are explicitly allowed to import multiple specific
symbols on one line from the \texttt{typing} and
\texttt{collections.abc} modules. For example:

\begin{samepage}
\begin{Shaded}
\begin{Highlighting}[]
\ImportTok{from}\NormalTok{ collections.abc }\ImportTok{import}\NormalTok{ Mapping, Sequence}
\ImportTok{from}\NormalTok{ typing }\ImportTok{import}\NormalTok{ Any, Generic, cast, TYPE\_CHECKING}
\end{Highlighting}
\end{Shaded}
\end{samepage}

Given that this way of importing adds items to the local namespace,
names in \texttt{typing} or \texttt{collections.abc} should be treated
similarly to keywords, and not be defined in your Python code, typed or
not. If there is a collision between a type and an existing name in a
module, import it using \texttt{import\ x\ as\ y}.

\begin{samepage}
\begin{Shaded}
\begin{Highlighting}[]
\ImportTok{from}\NormalTok{ typing }\ImportTok{import}\NormalTok{ Any }\ImportTok{as}\NormalTok{ AnyType}
\end{Highlighting}
\end{Shaded}
\end{samepage}

Prefer to use built-in types as annotations where available. Python
supports type annotations using parametric container types via
\href{https://peps.python.org/pep-0585/}{PEP-585}, introduced in Python
3.9.

\begin{samepage}
\begin{Shaded}
\begin{Highlighting}[]
\KeywordTok{def}\NormalTok{ generate\_foo\_scores(foo: }\BuiltInTok{set}\NormalTok{[}\BuiltInTok{str}\NormalTok{]) }\OperatorTok{{-}\textgreater{}} \BuiltInTok{list}\NormalTok{[}\BuiltInTok{float}\NormalTok{]:}
\NormalTok{  ...}
\end{Highlighting}
\end{Shaded}
\end{samepage}

\paragraph{Conditional Imports}

Use conditional imports only in exceptional cases where the additional
imports needed for type checking must be avoided at runtime. This
pattern is discouraged; alternatives such as refactoring the code to
allow top-level imports should be preferred.

Imports that are needed only for type annotations can be placed within
an \texttt{if\ TYPE\_CHECKING:} block.

\begin{itemize}
\tightlist
\item
  Conditionally imported types need to be referenced as strings, to be
  forward compatible with Python 3.6 where the annotation expressions
  are actually evaluated.
\item
  Only entities that are used solely for typing should be defined here;
  this includes aliases. Otherwise it will be a runtime error, as the
  module will not be imported at runtime.
\item
  The block should be right after all the normal imports.
\item
  There should be no empty lines in the typing imports list.
\item
  Sort this list as if it were a regular imports list.
\end{itemize}

\begin{samepage}
\begin{Shaded}
\begin{Highlighting}[]
\ImportTok{import}\NormalTok{ typing}
\ControlFlowTok{if}\NormalTok{ typing.TYPE\_CHECKING:}
  \ImportTok{import}\NormalTok{ sketch}
\KeywordTok{def}\NormalTok{ f(x: }\StringTok{"sketch.Sketch"}\NormalTok{): ...}
\end{Highlighting}
\end{Shaded}
\end{samepage}

\paragraph{Circular Dependencies}

Circular dependencies that are caused by typing are code smells. Such
code is a good candidate for refactoring. Although technically it is
possible to keep circular dependencies, various build systems will not
let you do so because each module has to depend on the other.

Replace modules that create circular dependency imports with
\texttt{Any}. Set an \hyperref[typing-aliases]{alias} with a meaningful
name, and use the real type name from this module (any attribute of
\texttt{Any} is \texttt{Any}). Alias definitions should be separated
from the last import by one line.

\begin{samepage}
\begin{Shaded}
\begin{Highlighting}[]
\ImportTok{from}\NormalTok{ typing }\ImportTok{import}\NormalTok{ Any}

\NormalTok{some\_mod }\OperatorTok{=}\NormalTok{ Any  }\CommentTok{\# some\_mod.py imports this module.}
\NormalTok{...}

\KeywordTok{def}\NormalTok{ my\_method(}\VariableTok{self}\NormalTok{, var: }\StringTok{"some\_mod.SomeType"}\NormalTok{) }\OperatorTok{{-}\textgreater{}} \VariableTok{None}\NormalTok{:}
\NormalTok{  ...}
\end{Highlighting}
\end{Shaded}
\end{samepage}

\paragraph{Generics}

When annotating, prefer to specify type parameters for generic types;
otherwise, \href{https://peps.python.org/pep-0484/\#the-any-type}{the
generics' parameters will be assumed to be \texttt{Any}}.

\begin{samepage}
\begin{Shaded}
\begin{Highlighting}[]
\CommentTok{\# Yes:}
\KeywordTok{def}\NormalTok{ get\_names(employee\_ids: Sequence[}\BuiltInTok{int}\NormalTok{]) }\OperatorTok{{-}\textgreater{}}\NormalTok{ Mapping[}\BuiltInTok{int}\NormalTok{, }\BuiltInTok{str}\NormalTok{]:}
\NormalTok{  ...}
\end{Highlighting}
\end{Shaded}
\end{samepage}

\begin{samepage}
\begin{Shaded}
\begin{Highlighting}[]
\CommentTok{\# No:}
\CommentTok{\# This is interpreted as get\_names(employee\_ids: Sequence[Any]) {-}\textgreater{} Mapping[Any, Any]}
\KeywordTok{def}\NormalTok{ get\_names(employee\_ids: Sequence) }\OperatorTok{{-}\textgreater{}}\NormalTok{ Mapping:}
\NormalTok{  ...}
\end{Highlighting}
\end{Shaded}
\end{samepage}

If the best type parameter for a generic is \texttt{Any}, make it
explicit, but remember that in many cases
\hyperref[typing-type-var]{\texttt{TypeVar}} might be more appropriate:

\begin{samepage}
\begin{Shaded}
\begin{Highlighting}[]
\CommentTok{\# No:}
\KeywordTok{def}\NormalTok{ get\_names(employee\_ids: Sequence[Any]) }\OperatorTok{{-}\textgreater{}}\NormalTok{ Mapping[Any, }\BuiltInTok{str}\NormalTok{]:}
  \CommentTok{"""Returns a mapping from employee ID to employee name for given IDs."""}
\end{Highlighting}
\end{Shaded}
\end{samepage}

\begin{samepage}
\begin{Shaded}
\begin{Highlighting}[]
\CommentTok{\# Yes:}
\NormalTok{\_T }\OperatorTok{=}\NormalTok{ TypeVar(}\StringTok{\textquotesingle{}\_T\textquotesingle{}}\NormalTok{)}
\KeywordTok{def}\NormalTok{ get\_names(employee\_ids: Sequence[\_T]) }\OperatorTok{{-}\textgreater{}}\NormalTok{ Mapping[\_T, }\BuiltInTok{str}\NormalTok{]:}
  \CommentTok{"""Returns a mapping from employee ID to employee name for given IDs."""}
\end{Highlighting}
\end{Shaded}
\end{samepage}

\section{Parting Words}
\emph{BE CONSISTENT}.

If you're editing code, take a few minutes to look at the code around
you and determine its style. If they use \texttt{\_idx} suffixes in
index variable names, you should too. If their comments have little
boxes of hash marks around them, make your comments have little boxes of
hash marks around them too.

The point of having style guidelines is to have a common vocabulary of
coding so people can concentrate on what you're saying rather than on
how you're saying it. We present global style rules here so people know
the vocabulary, but local style is also important. If code you add to a
file looks drastically different from the existing code around it, it
throws readers out of their rhythm when they go to read it.

However, there are limits to consistency. It applies more heavily
locally and on choices unspecified by the global style. Consistency
should not generally be used as a justification to do things in an old
style without considering the benefits of the new style, or the tendency
of the codebase to converge on newer styles over time.

\end{document}
